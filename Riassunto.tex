\documentclass[a4paper,12pt,italian]{article}
\usepackage[italian]{babel}
\usepackage[utf8]{inputenc}
\usepackage{graphicx}
\usepackage{wrapfig}
\usepackage{amsmath}
\usepackage{amssymb}
\usepackage{appendix}
\usepackage{amscd}
\usepackage{subfig}
\usepackage{cite}
\usepackage[autostyle,italian=guillemets]{csquotes}
\usepackage{bm}
\usepackage{vmargin}

\setmarginsrb{30mm}{30mm}{30mm}{20mm}{0pt}{0mm}{0pt}{0mm}
\hyphenation{ShuntLDO}
\hyphenation{DC-DC}

\begin{document}
%
%---- SLASH
\def\slasha#1{#1\hskip-0.65em /}  %slasha per caratteri piccoli
\def\slashb#1{#1\hskip-1.3em /}   %slashb per quelli grandi
\def\slashc#1{#1\hskip-.4em /}
%
%---- UNITA` DI MISURA
\def \pb        {{\rm \, pb}}
\def \ipb       {{\rm \, pb^{-1}}}
\def \ifb       {{\rm \, fb^{-1}}}
\def \eV        {{\rm \,  e\kern-0.125em V}}
\def \keV       {{\rm \, ke\kern-0.125em V}}
\def \MeV       {{\rm \, Me\kern-0.125em V}}
\def \GeV       {{\rm \, Ge\kern-0.125em V}}
\def \TeV       {{\rm \, Te\kern-0.125em V}}
\def \Hz       {{\rm \, Hz}}
\def \kHz       {{\rm \, kHz}}
\def \MHz       {{\rm \, MHz}}
\def \GHz       {{\rm \, GHz}}
\def \Ohm       {{\rm \, \Omega}}
\def \mOhm      {{\rm \, m\Omega}}
\def \kOhm      {{\rm \, k\Omega}}
\def \MGy       {{\rm \, MGy}}
\def \Grad      {{\rm \, Mrad}}
\def \TeVc      {\TeV\kern-0.125em /c}
\def \TeVcc     {\TeV\kern-0.125em /c^2}
\def \GeVc      {\GeV\kern-0.125em /c}
\def \GeVcc     {\GeV\kern-0.125em /c^2}
\def \MeVc      {\MeV\kern-0.125em /c}
\def \MeVcc     {\MeV\kern-0.125em /c^2}
%
%---- SIMBOLI
\def\ga{\mathrel{\raise.3ex\hbox{$>$\kern-.75em\lower1ex\hbox{$\sim$}}}}
\def\la{\mathrel{\raise.3ex\hbox{$<$\kern-.75em\lower1ex\hbox{$\sim$}}}}
%\newcommand {\lesssim}
%     {\,\raisebox{-0.6ex}{$\stackrel{\textstyle<}{\textstyle\sim}$}\,}
%\newcommand {\gtrsim}
%     {\,\raisebox{-0.6ex}{$\stackrel{\textstyle>}{\textstyle\sim}$}\,}
\newcommand{\ckm}{$\checkmark$}
%
%---- MISCELLANEA
\newcommand {\slashed}[1] { \mbox{\rlap{\hbox{/}} #1 }}
\newcommand {\onehalf}    {\raisebox{0.1ex}{${\frac{1}{2}}$}}
\newcommand {\fivethirds} {\raisebox{0.1ex}{${\frac{5}{3}}$}}
\newcommand {\OR}         {{\tt OR}\,}
\newcommand {\rts}        {\sqrt{s}}
\newcommand {\lumi}       {\mathcal{L}}
\newcommand {\Lumi}       {\int\lumi\mathrm{d}t}
\newcommand {\degrees}    {^\circ}
\newcommand {\VDDD}       {\mathrm{V_{outD}}}
\newcommand {\VDDA}       {\mathrm{V_{outA}}}
\newcommand {\Iin}        {\mathrm{I_{in}}}
\newcommand {\Vin}        {\mathrm{V_{in}}}
\newcommand {\Iload}      {\mathrm{I_{load}}}
\newcommand {\de}         {\partial}
\newcommand {\uA}         {\; \mu \rm A}
\newcommand {\um}         {\; \mu \rm m}
\newcommand {\nm}         {\rm \; nm}
\newcommand {\ms}         {\; \rm m \rm s}
\newcommand {\us}         {\; \mu \rm s}
\newcommand {\cm}         {\rm \; cm}
\newcommand {\mm}         {\rm \; mm}
\newcommand {\mV}         {\rm \; mV}
\newcommand {\mA}         {\rm \; mA}
\newcommand {\m}          {\rm \; m}
\newcommand {\km}         {\rm \; km}
\newcommand {\bx}         {\rm \; BX}
\newcommand {\A}          {\rm \, A}
\newcommand {\V}          {\rm \; V}
\newcommand {\K}          {\rm \; V}
\newcommand {\T}          {\rm \; T}
\newcommand {\kV}         {\rm \; kV}
\newcommand {\kW}         {\rm \; kW}
\newcommand {\kA}         {\rm \; kA}
\newcommand {\kVm}        {\rm \; kV\! / \! m} 
\newcommand {\MVm}        {\rm \; MV\! / \! m} 
\newcommand {\ns}         {\rm \; ns} 
%
%---- THEORY groups & AOB
\newcommand {\gws}        {\mathrm{SU(2)_L \otimes U(1)_Y}}
\newcommand {\sul}        {\mathrm{SU(2)_L}}
\newcommand {\suc}        {\mathrm{SU(3)_C}}
\newcommand {\ul}         {\mathrm{U(1)_Y}}
\newcommand {\uem}        {\mathrm{U(1)_{em}}}
\newcommand {\sigmabar}   {\overline{\sigma}}
\newcommand {\gmunu}      {g^{\mu \nu}}
\newcommand {\munu}       {{\mu \nu}}
\newcommand {\obra}       {\langle 0 |}
\newcommand {\oket}       {| 0 \rangle}
%
%---- THEORY lepton fields
\newcommand {\LL}         {L^{\alpha}_{\mathrm L}}
\newcommand {\LLd}        {L^{\dagger \alpha}_{\mathrm L}}
\newcommand {\lL}         {\ell^{\alpha}_{\mathrm L}}
\newcommand {\lLd}        {\ell^{\dagger \alpha}_{\mathrm L}}
\newcommand {\ld}         {\ell^{\dagger \alpha}}
\newcommand {\lb}         {\overline{\ell}^{\alpha}}
\newcommand {\lR}         {\ell^{\alpha}_{\mathrm R}}
\newcommand {\lRd}        {\ell^{\dagger \alpha}_{\mathrm R}}
\newcommand {\nuL}        {\nu^{\alpha}_{\mathrm L}}
\newcommand {\nuLb}       {\overline{\nu}^{\alpha}_{\mathrm L}}
\newcommand {\nub}        {\overline{\nu}^{\alpha}}
\newcommand {\lept}       {\ell^\alpha}
\newcommand {\neut}       {\nu^{\alpha}}
\newcommand {\nuLd}       {\nu^{\dagger \alpha}_{\mathrm L}}
\newcommand {\Phid}       {\Phi^\dagger}
%
%---- THEORY quark fields
\newcommand {\up}         {u^{\alpha}}
\newcommand {\ub}         {\overline{u}^{\alpha}}
\newcommand {\down}       {d^{\alpha}}
\newcommand {\db}         {\overline{d}^{\alpha}}
\newcommand {\QL}         {Q^{\alpha}_{\mathrm L}}
\newcommand {\QLd}        {Q^{\dagger \alpha}_{\mathrm L}}
\newcommand {\UL}         {U^{\alpha}_{\mathrm L}}
\newcommand {\ULd}        {U^{\dagger \alpha}_{\mathrm L}}
\newcommand {\UR}         {U^{\alpha}_{\mathrm R}}
\newcommand {\URd}        {U^{\dagger \alpha}_{\mathrm R}}
\newcommand {\DL}         {D^{\alpha}_{\mathrm L}}
\newcommand {\DLd}        {D^{\dagger \alpha}_{\mathrm L}}
\newcommand {\DR}         {D^{\alpha}_{\mathrm R}}
\newcommand {\DRd}        {D^{\dagger \alpha}_{\mathrm R}}
\newcommand {\bfell}      {\ell\kern-0.4em
                           \ell\kern-0.4em
                           \ell\kern-0.4em
                           \ell }
\newcommand {\obfell}     {\overline{\ell}\kern-0.4em
                           \overline{\ell}\kern-0.4em
                           \overline{\ell}\kern-0.4em
                           \overline{\ell}}
\newcommand {\bfH}      {\; {\cal H}\kern-0.5em \kern-0.4em
                           {\cal H}\kern-0.5em \kern-0.4em
                           {\cal H}\kern0.1em }
\newcommand {\obfH}     {\; \overline{\cal H}\kern-0.5em \kern-0.4em 
                           \overline{\cal H}\kern-0.5em \kern-0.4em 
                           \overline{\cal H}\kern0.1em }
%
%---- PARTICELLE
\def \b             {{\mathrm b}}
\def \t             {{\mathrm t}}
\def \c             {{\mathrm c}}
\def \d             {{\mathrm d}}
\def \u             {{\mathrm u}}
\def \e             {{\mathrm e}}
\def \q             {{\mathrm q}}
\def \g             {{\mathrm g}}
\def \p             {{\mathrm p}}
\def \s             {{\mathrm s}}
\def \n             {{\mathrm n}}
\def \l             {\ell} 
%\def \f             {{\mathrm f}} 
\def \f             {{f}} 
\def \D             {{\mathrm D}}
\def \K             {{\mathrm K}}
\def \Z             {{\mathrm Z}}
\def \W             {{\mathrm W}}
\def \S             {{\mathrm S}}
\def \N             {{\mathrm N}}
\def \L             {{\mathrm L}}
\def \R             {{\mathrm R}}
%
%---- SUSY
\newcommand {\dm}         {\Delta m}
\newcommand {\dM}         {\Delta M}
\newcommand {\ldm}        {\mbox{``low $\dm$''}}
\newcommand {\hdm}        {\mbox{``high $\dm$''}}
\newcommand {\nnc}        {{\overline{\mathrm n}_{95}}}
\newcommand {\snc}        {{\overline{\sigma}_{95}}}
\newcommand {\susy}       {{supersymmetry}}
\newcommand {\susyc}      {{supersymmetric}}
\newcommand {\aj}         {\mbox{\sf AJ}}
\newcommand {\ajl}        {\mbox{\sf AJL}}
\newcommand {\llh}        {\mbox{\sf LLH}}
%
%---- SPARTICELLE
\newcommand {\rpc}     {{\rm RPC}}
\newcommand {\rpv}     {{\rm RPV}}
\newcommand {\sfe}     {{\tilde{f}}}
\newcommand {\sfL}     {{\tilde{f}_{\mathrm L}}}
\newcommand {\sfR}     {{\tilde{f}_{\mathrm R}}}
\newcommand {\sfone}   {{\tilde{f}_{1}}}
\newcommand {\sftwo}   {{\tilde{f}_{2}}}
\newcommand {\sneu}    {{\tilde{\nu}}}
\newcommand {\wino}    {{\mathrm{\widetilde{W}}}}
\newcommand {\bino}    {{\mathrm{\widetilde{B}}}}
\newcommand {\se}      {{\mathrm{\tilde{e}}}}
\newcommand {\seR}     {{\mathrm{\tilde{e}_{R}}}}
\newcommand {\seL}     {{\mathrm{\tilde{e}_{L}}}}
\newcommand {\st}      {{\mathrm{\tilde{\tau}}}}
\newcommand {\stR}     {{\mathrm{\tilde{\tau}_{R}}}}
\newcommand {\stL}     {{\mathrm{\tilde{\tau}_{L}}}}
\newcommand {\stone}   {{\mathrm{\tilde{\tau}_{1}}}}
\newcommand {\sttwo}   {{\mathrm{\tilde{\tau}_{2}}}}
\newcommand {\sm}      {{\mathrm{\tilde{\mu}}}}
\newcommand {\smR}     {{\mathrm{\tilde{\mu}_{R}}}}
\newcommand {\smL}     {{\mathrm{\tilde{\mu}_{L}}}}
\newcommand {\Sup}     {{\mathrm{\tilde{u}}}}
\newcommand {\suR}     {{\mathrm{\tilde{u}_{R}}}}
\newcommand {\suL}     {{\mathrm{\tilde{u}_{L}}}}
\newcommand {\sdo}     {{\mathrm{\tilde{d}}}}
\newcommand {\sdR}     {{\mathrm{\tilde{d}_{R}}}}
\newcommand {\sdL}     {{\mathrm{\tilde{d}_{L}}}}
\newcommand {\sch}     {{\mathrm{\tilde{c}}}}
\newcommand {\scR}     {{\mathrm{\tilde{c}_{R}}}}
\newcommand {\scL}     {{\mathrm{\tilde{c}_{L}}}}
\newcommand {\sst}     {{\mathrm{\tilde{s}}}}
\newcommand {\ssR}     {{\mathrm{\tilde{s}_{R}}}}
\newcommand {\ssL}     {{\mathrm{\tilde{s}_{L}}}}
\newcommand {\stopR}   {{\mathrm{\tilde{t}_{R}}}}
\newcommand {\stopL}   {{\mathrm{\tilde{t}_{L}}}}
\newcommand {\stopone} {{\mathrm{\tilde{t}_{1}}}}
\newcommand {\stoptwo} {{\mathrm{\tilde{t}_{2}}}}
\newcommand {\sto}     {{\mathrm{\tilde{t}}}}
\newcommand {\SQ}      {{\mathrm{\widetilde{Q}}}}
\newcommand {\STO}     {{\mathrm{\widetilde{T}}}}
\newcommand {\glu}     {{\mathrm{\tilde{g}}}}
\newcommand {\sbotR}   {{\mathrm{\tilde{b}_{R}}}}
\newcommand {\sbotL}   {{\mathrm{\tilde{b}_{L}}}}
\newcommand {\sbotone} {{\mathrm{\tilde{b}_{1}}}}
\newcommand {\sbottwo} {{\mathrm{\tilde{b}_{2}}}}
\newcommand {\sbot}    {{\mathrm{\tilde{b}}}}
\newcommand {\squa}    {{\tilde{\mathrm{q}}}}
\newcommand {\squal}   {{\tilde{\mathrm{q}}_{\rm L}}}
\newcommand {\squar}   {{\tilde{\mathrm{q}}_{\rm R}}}
\newcommand {\sqL}     {{\tilde{\mathrm{q}}_{\rm L}}}
\newcommand {\sqR}     {{\tilde{\mathrm{q}}_{\rm R}}}
\newcommand {\snu}     {{\tilde{\nu}}}
\newcommand {\snue}    {{\tilde{\nu}_{\mathrm e}}}
\newcommand {\snum}    {{\tilde{\nu}_{\mu}}}
\newcommand {\snut}    {{\tilde{\nu}_{\tau}}}
\newcommand {\neu}     {{\chi}}
\newcommand {\chap}    {{\chi^+}}
\newcommand {\cham}    {{\chi^-}}
\newcommand {\chapm}   {{\chi^\pm}}

%
%---- SUSY PARAMETRI
\newcommand {\thstop} {\mathrm{\theta_{\tilde{t}}}}
\newcommand {\thsbot} {\mathrm{\theta_{\tilde{b}}}}
\newcommand {\thsqua} {\mathrm{\theta_{\tilde{q}}}}
\newcommand {\Mcha}{M_{\chi^\pm}}
\newcommand {\Mchi}{M_\chi}
\newcommand {\Msnu}{M_{\tilde{\nu}}}
\newcommand {\tanb}{\tan\beta}
%
%---- ABBREVIAZIONI

%
%---- PROCESSI FISICI
\newcommand {\rb}    {{\rm R_{\b}}}
\newcommand {\qq}    {{\q \overline{\q}}}
\newcommand {\bb}    {{\b \overline{\b}}}
\newcommand {\ff}    {{\f \bar{\f}}}
\newcommand {\el}    {{\e ^+}}
\newcommand {\po}    {{\e ^-}}
\newcommand {\ee}    {{\e ^+ \e ^-}}
\newcommand {\gaga}  {\gamma\gamma}
\newcommand {\ggqq}  {\gamma\gamma \rightarrow \q\overline{\q}}
\newcommand {\ggtt}  {\gamma\gamma \rightarrow \tau^{+}\tau^{-}}
\newcommand {\qqg}   {\q\overline{\q}\gamma}
\newcommand {\ttg}   {\tau^{+}\tau^{-}\gamma}
\newcommand {\wenu}  {{\rm We\nu_\e}}
\newcommand {\gsZ}   {\gamma^\star\mathrm{Z}}
\newcommand {\ggh}   {\gamma\gamma\rightarrow{\mathrm{hadrons}}}
\newcommand {\ZZg}   {\mathrm ZZ^{*}/\gamma^{*}}
%
%---- VARIABILI
\newcommand {\zo}      {{z_0}}
\newcommand {\ip}      {{d_0}}
\newcommand {\thr}     {{T_{\rm thrust}}}
\newcommand {\athr}    {{\hat{\rm a}_{\rm thrust}}}
\newcommand {\acol}    {{\Phi_{\rm acol}}}
\newcommand {\acop}    {{\Phi_{\rm acop}}}
\newcommand {\acopt}   {{\Phi_{\rm acop_T}}}
\newcommand {\thpoint} {\theta_{\rm point}}
\newcommand {\thscat}  {\theta_{\rm scat}}
\newcommand {\etwelve} {E^{\, \bowtie  12\degrees}_z}
\newcommand {\ethirty} {E^{\, \bowtie  30\degrees}_z}
\newcommand {\eiso}[1] {E^{\, \triangleleft 30\degrees}_{#1}}
\newcommand {\ewedge}  {E(\phi_{\vec{p}_{\rm miss}}\pm 15\degrees)}
\newcommand {\evis}    {E_{\rm vis}}
\newcommand {\emis}    {E_{\rm miss}}
\newcommand {\mvis}    {M_{\rm vis}}
\newcommand {\mmis}    {M_{\rm miss}}
\newcommand {\mhad}    {M^{\rm ex \, \ell_1}_{\rm vis}}
\newcommand {\mhadtwo} {M^{\rm ex \, \ell_1\ell_2}_{\rm vis}}
\newcommand {\ehad}    {E^{\rm NH}_{\rm vis}}
\newcommand {\epho}    {E^{\gamma}_{\rm vis}}
\newcommand {\echa}    {E^{\rm ch}_{\rm vis}}
\newcommand {\nch}     {{N_{\rm ch}}}
\newcommand {\elept}   {E_{\rm lept}}
\newcommand {\elepone} {E_{\ell\ 1}}
\newcommand {\pvis}    {{\vec{p}_{\rm vis}}}
\newcommand {\pmis}    {{\vec{p}_{\rm miss}}}
\newcommand {\pt}      {{p_{\rm t}}}
\newcommand {\ptch}    {{p_{\rm t}^{\rm ch}}}
\newcommand {\pz}      {{p_z}}
\newcommand {\ptnoNH}  {{p_{\rm t}^{\rm ex \, NH}}}
\newcommand {\puds}    {{P_{\rm uds}}}
%
%
% no more of Christian's random capitalization!
% more of mine
\newcommand{\brchal}{\cal{B}($\PCha \rightarrow \ell\nu\PChi\ $)}
\newcommand{\M}{M_{2}}
\newcommand{\Mp}{M_{2}}
\newcommand{\sigbg}{\sigma_{\mathrm{bg}}}
\newcommand{\ww}   {\mathrm {WW}}
\newcommand{\zz}   {\mathrm Z\gamma^{*}}
\newcommand{\ewnu} {\mathrm{eW}\nu}
\newcommand{\eez}  {\mathrm {eeZ}}
\newcommand{\gagall}{{\gamma\gamma\rightarrow \ell\ell }}
\newcommand{\Pstaup}{{\widetilde{\tau}_{1}}}
\newcommand{\Pstaul}{{\widetilde{\tau}_{L}}}
\newcommand{\Pstaur}{{\widetilde{\tau}_{R}}}
\newcommand{\mzero}{m_{0}}
\newcommand{\msnu}{M_{\tilde{\nu}}}
\newcommand{\mcha}{M_{\chi^{\pm}}}
\newcommand{\mchi}{M_{\chi}}
\newcommand{\mstau}{M_{{\widetilde{\tau}_{1}}}}
\newcommand{\atau}{A_{\tau}}
\newcommand{\chsnu}{\PCha \rightarrow \ell \tilde{\nu}}
\newcommand{\chstau}{\PCha \rightarrow \tilde{\tau}_{1}\nu}
\newcommand{\chlep}{\PCha \rightarrow \ell\nu\chi}
\newcommand{\Tcsq}{\mathrm{TeV}/c^2}
% new for thesis
\newcommand{\nobs}{N_{\mathrm{obs}}}
\newcommand{\nlim}{N_{\mathrm{lim}}}
\newcommand{\Brl}{\cal{B}_{\ell}}
\newcommand{\leff} {\mathcal{L}_{\mathrm{eff}}}
\newcommand{\dedx}{{\mathrm{d}}E/{\mathrm{d}}x}
\newcommand{\chtau}{\PCha \rightarrow \tau\nu\chi}
\newcommand{\ssqtw}{\sin^{2}\theta_{\mathrm W}}
%\newcommand{\PSql}{\tilde{\mathrm q}_L}
%\newcommand{\PSqr}{\tilde{\mathrm q}_R}
%\newcommand{\PSq1}{\tilde{\mathrm q}_1}
%\newcommand{\PSq2}{\tilde{\mathrm q}_2}
%\newcommand{\ww}{{\mathrm WW}}
%\newcommand{\zz}{{\mathrm Z\gamma^{*}}}
%\newcommand{\eez}{{\mathrm eeZ}}
\newcommand{\nnz}{{\mathrm \nu\bar{\nu}Z}}
% added by bill
\def \ggll    {\gamma\gamma \rightarrow \ell^{+}{\ell}^{-}}
\def \tautau  {\mathrm \tau^{+}\tau^{-}}
\def \ffg  {f\bar{f}(\gamma)}
\def \lll   {\ell^{+}{\ell}^{-}}
\def \ww   {\mathrm WW}
\def \zz   {\mathrm Z\gamma^{*}}
\def \znn  {\mathrm Z\nu\nu}
\def \zee  {\mathrm Zee}
\def \rts  {\sqrt{s}}
\def \mstop {m_{\tilde{\mathrm{t}}}}
\def \msnu  {m_{\tilde{\nu}}}
\def \elow   {E_{12^{\circ}}}
\def \thmiss {\theta_{P_{\mathrm{miss}}}}
\def \gev    { \, \mathrm{GeV}/\it{c}^{\mathrm{2}}}
\def \gvm    { \, \mathrm{GeV}/\it{c}}
\def \mx     {M_{\mathrm{eff}}} 
\newcommand{\neutr}{\chi}
%end fabio



%dalla mia pretesi

%\def \X             {\mathrm X} 
%\def \V             {\mathrm V} 
\def \Zcc           {\Z \to \c \bar{\c} }
\def \Zbb           {\Z \to \b \bar{\b} }
\def \decDS         {\D^{*+} \to \D^0 \pi^+}
\def \decsDS        {\D^{*+} \to \D^0 \pi^+_s}
\def \deckp         {\D^{0} \to \K^- \pi^+}
\def \deckppp       {\D^{0} \to \K^- \pi^+ \pi^+ \pi^-}
\def \deckpp        {\D^{0} \to \K^- \pi^+ \pi^0}
\def \deckpS        {\D^{0} \to \K^- \pi^+ (\pi^0)}
\def \decskp        {\D^{*+} \to \pi^{+}_{s} \K^- \pi^+}
\def \decskppp      {\D^{*+} \to \pi^{+}_{s} \K^- \pi^+ \pi^+ \pi^-}
\def \decskpp       {\D^{*+} \to \pi^{+}_{s} \K^- \pi^+ \pi^0}
\def \decskpS       {\D^{*+} \to \pi^{+}_{s} \K^- \pi^+ (\pi^0)}
\def \epsc          {\varepsilon_{\c}}
\def \epsb          {\varepsilon_{\b}}
\def \pctod         {P_{\c \to \D^*}}
\def \pbtod         {P_{\b \to \D^*}}
%\def \R             {{\mathrm R}}
\def \Gbb           {\Gamma_{\b\bar{\b}}}
\def \Gcc           {\Gamma_{\c\bar{\c}}}
\def \Gh            {\Gamma_{\mathrm h}}





\pagestyle{plain}
\pagenumbering{gobble}
\noindent

%\begin{center}
\noindent {\em Relatore} {\bf Dott. Giacomo Sguazzoni}, {\tt giacomo.sguazzoni@fi.infn.it}\\
\noindent {\em Correlatore} {\bf Prof. Raffaello D'Alessandro}, {\tt raffaello.dalessandro@unifi.it}\\
\noindent {\em Candidato} {\bf Andrea Fiaschi}, {\tt andrea.fiaschi2@stud.unifi.it}

\vskip 0.5cm

\noindent \textbf{Sviluppo di un sistema innovativo di alimentazione del tracciatore interno di CMS per la fase ad alta luminosità di LHC}

\vskip 0.5cm

 L'alimentazione seriale \`e una tecnologia di frontiera, mai utilizzata prima d'ora in un esperimento di fisica delle alte energie, ma in fase di sviluppo per l'Inner Tracker, il rivelatore a pixel del nuovo Tracciatore di CMS per HL-LHC. A causa delle scelte progettuali dettate dall'elevatissima luminosit\`a istantanea ($\mathrm{5-7.5\cdot 10^{34}cm^{-2}s^{-1}}$) e dalla radiazione (fino a $\mathrm{2.3 [1\MeV neq]/cm^2}$, $1.2\Grad$), il funzionamento dei due miliardi di canali dell'Inner Tracker richiede $50-60\kW$ di potenza che, alimentando i moduli in catene di una decina elementi, possono essere trasportati riducendo drasticamente il materiale passivo associato ai cavi. Questo \`e un aspetto cruciale per il successo del nuovo rivelatore dedicato all'ambizioso programma di HL-LHC, focalizzato sulle misure di precisione nel settore di Higgs e sulle ricerche di nuova fisica oltre il Modello Standard.

Il cuore dello schema di alimentazione seriale \`e lo ShuntLDO, uno speciale circuito con un regolatore \textit{Low Drop Out} accoppiato ad uno shunt. 

In questo lavoro di tesi, svolto tra l'INFN, Sezione di Firenze, e il CERN, Ginevra, Svizzera, ho caratterizzato i prototipi di ShuntLDO realizzati in tecnologia CMOS a $65\nm$, con dimensionamento da $0.5\A$ e $2\A$, e lo ShuntLDO presente sul primo prototipo ROC (il \textit{Readout Chip} dell'elettronica di prossimit\`a), RD53A, sviluppato per HL-LHC dalla collaborazione RD53. 

Lo ShuntLDO garantisce che il singolo elemento nella catena seriale si presenti come un carico costante essenzialmente resistivo, pi\`u un offset di tensione in serie, indipendentemente dal consumo istantaneo. Gli eventuali picchi di corrente del ROC sono gestiti localmente.
% meno di un a caratterizzato da un offset $\mathrm{V_{offset}}$ ed una resistenza caratteristica R.
Grazie a queste peculiarit\`a gli ShuntLDO possono essere posti in parallelo senza controindicazioni, aspetto chiave per un rivelatore di grande scala e complessit\`a in cui i moduli pixel verranno posti in serie ma, su ognuno di questi vi saranno più ROC alimentati in parallelo.
 
%Pi\`u ShuntLDO possono eQueste caratteristiche permettono la massima flessibilit\`a permette di mettere più elementi in parallelo.
%a distribuzione di corrente tra i vari elementi dipenderà dal parallelo delle varie R caratteristiche.

Nel lavoro di tesi ho studiato il comportamento di questi dispositivi nelle varie versioni sia in condizioni stazionarie con carichi statici, sia con carichi dinamici per verificare l'effettiva capacità dello ShuntLDO di isolare il ROC rispetto alla linea di alimentazione e viceversa. 

Grazie ai risultati ottenuti sono stato in grado di confermare l'affidabilità del principio di funzionamento alla base del progetto dello ShuntLDO, dimostrandone la capacità di isolare il ROC da influenze esterne anche in situazioni limite e evidenziando gli aspetti che possono risultare critici nel sistema finale quali, ad esempio, l'interazione tra lo ShuntLDO e la circuiteria ancillare necessaria per il suo funzionamento specialmente nella fase di accensione.

I risultati che ho ottenuto non evidenziano criticit\`a di questa innovativa tecnologia e permettono di proseguire lo sviluppo per giungere al progetto finale.

\end{document}


cruciali per avere un bilanciamento delle correnti durante l'accensione e 
presenti sulla linea di alimentazione. 
Allo stesso modo ho evidenziato criticità e problemi dell'implementazione dello ShuntLDO all'interno del ROC RD53A, tutt'ora sotto studio. Tra queste spiccano l'importanza di un circuito che generi le tensioni di riferimento in modo affidabile, la presenza di resistenze spurie e una dipendenza del valore dell'offset dal valore della resistenza caratteristica.

 Nonostante l'impiego di tecnologia CMOS a $65\nm$ con alimentazione ${\cal O}(1\V)$, il consumo di potenza dell'elettronica di prossimit\`a \`e enorme a causa dell'elevato rate ($3\GHz/\mathrm{cm^2}$), dovuto alla luminosit\`a istantanea ($\mathrm{5-7.5\cdot 10^{34}cm^{-2}s^{-1}}$), e il basso rumore, indispensabile per le basse soglie necessarie per mitigare gli effetti del danneggiamento da radiazione (fino a $\mathrm{2.3 [1\MeV neq]/cm^2}$, $1.2\Grad$). L'alimentazione seriale \`e l'unico schema possibile per il funzionamento dei due miliardi di canali dell'Inner Tracker perché consente di trasportare i $50-60\kW$ di potenza, necessari al funzionamento, ad una tensione pi\`u elevata, ${\cal O}(10\V)$, concatenando una decina di moduli a pixel.

Altri dispositivi di conversione locale, come ad esempio i convertitori DC-DC, non possono essere utilizzati a causa della elevatissima radiazione e del ridotto spazio disponibile nel rivelatore, un cilindro di $\sim5\m$ di lunghezza intorno alla \textit{beam pipe} con un raggio compreso tra $\sim20$ e $\sim 30\cm$. L'alimentazione diretta richiederebbe, invece, un materiale passivo, in cavi, circa dieci volte maggiore. 
Ciò vanificherebbe i miglioramenti di prestazioni che questo nuovo rivelatore \`e chiamato ad avere per l'ambizioso programma di HL-LHC, focalizzato sulle misure di precisione nel settore di Higgs e sulle ricerche di nuova fisica oltre il Modello Standard. 

%In questo lavoro di tesi è stato studiato il circuito che sarà utilizzato localmente sui ROC (\textit{Read Out Chip}) per gestire l'alimentazione seriale che è prevista per il tracciatore di CMS di fase-2 in vista della fase ad alta luminosità del \textit{Large Hadron Collider} (LHC), chiamata HL-LHC (\textit{High Luminosity-LHC}). 
In questo lavoro di tesi ho studiato il circuito che sarà utilizzato localmente sui ROC (\textit{Read Out Chip}) per la gestione dell'alimentazione seriale prevista per il tracciatore di CMS di fase-2, in vista della fase ad alta luminosità del \textit{Large Hadron Collider} (LHC), chiamata HL-LHC (\textit{High Luminosity-LHC}). 
Il sistema che gestisce le tensioni PoL (\textit{Point of Load}) sfrutta un circuito innovativo, in tecnologia CMOS a $65 \nm$, basato su un regolatore con caratteristiche di LDO (\textit{Low Drop Out}) accoppiato ad uno Shunt, o ShuntLDO. 
La scelta di utilizzare un'alimentazione seriale per il tracciatore a pixel di fase-2 è una novità rispetto al tracciatore di fase-0 e fase-1.
Il motivo di questa scelta risiede nella necessità di minimizzare la sezione dei cavi all'interno del rivelatore, che sono materiale inerte e causa di un peggioramento delle prestazioni. 
L'altro possibile approccio, analogo a quello utilizzato per il tracciatore a pixel di fase-1, consiste nell'utilizzo di convertitori DC-DC. L'impiego di questa soluzione è però afflitta da due limitazioni importanti che ne impediscono l'implementazione. 
Il primo motivo è che, essendo dispositivi di potenza, la resistenza alla radiazione degli ASIC su cui si basano i DC-DC è limitata e, dunque, dovrebbero essere posizionati a R$\sim$20--$25 \cm$ dall'asse dei fasci, richiedendo l'utilizzo di cavi di opportuna sezione per la trasmissione di potenza negli ultimi 10--$100 \cm$, vanificando, così, la riduzione di materiale in una regione molto critica. 
Il secondo motivo è la scarsa possibilità di miniaturizzazione dei DC-DC. 

Il lavoro di tesi si sviluppa a partire dallo studio dei prototipi di ShuntLDO da 0.5 e $2 \A$ per arrivare alla valutazione dello ShuntLDO presente nel prototipo del ROC, RD53A. 
Caratteristica fondamentale dello ShuntLDO è la capacità del circuito di alimentare un carico, nel nostro caso RD53A, gestendo localmente le variazioni e presentandosi alla catena seriale come un carico statico caratterizzato da un offset $\mathrm{V_{offset}}$ ed una resistenza caratteristica R. 
Questa caratteristica consente di alimentare la catena seriale con una corrente costante, dal momento che sarà lo ShuntLDO a regolare localmente le fluttuazioni del carico, e permette di mettere più elementi in parallelo.
La distribuzione di corrente tra i vari elementi dipenderà dal parallelo delle varie R caratteristiche. 
Nello studio di questi dispositivi mi sono concentrato sulla caratterizzazione della risposta con un carico statico, prima, e dinamico, poi, al variare della corrente di alimentazione e sulla verifica dell'effettiva capacità dello ShuntLDO di isolare il carico dai disturbi sulla linea e viceversa. 

Nel lavoro di tesi sono stato in grado di confermare l'affidabilità del principio
di funzionamento con cui è stato progettato lo ShuntLDO, evidenziando
gli aspetti cruciali per avere un bilanciamento delle correnti durante l'accensione e dimostrando la capacità dello ShuntLDO di isolare il carico dai disturbi esterni 
presenti sulla linea di alimentazione. 
Allo stesso modo ho evidenziato criticità e problemi dell'implementazione dello ShuntLDO all'interno del ROC RD53A, tutt'ora sotto studio. Tra queste spiccano l'importanza di un circuito che generi le tensioni di riferimento in modo affidabile, la presenza di resistenze spurie e una dipendenza del valore dell'offset dal valore della resistenza caratteristica.

