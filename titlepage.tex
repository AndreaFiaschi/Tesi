%\frontmatter

\begin{titlepage}
  %\vspace*{0.5cm}
  \begin{center}
    \begin{minipage}{\textwidth}
      \noindent\begin{minipage}{0.4\textwidth}%
      \includegraphics[width=1.\textwidth]{Immagini/logo}%
    \end{minipage}%
    \hfill
    \noindent\begin{minipage}{0.4\textwidth}
      \begin{flushright}
      {\bf
        Scuola di \\
        Scienze Matematiche \\ 
        Fisiche e Naturali \\
      }
      Corso di Laurea magistrale in Fisica Nucleare e Subnucleare\\
      
      \end{flushright}
    \end{minipage}
  \end{minipage}
  \vspace*{1.5cm}
  \vfill
  \begin{minipage}{0.8\textwidth}
  \begin{flushleft}
      {\large {\bf Sviluppo di un sistema innovativo di}} \\
      \vspace*{0.2cm}
      {\large {\bf alimentazione del tracciatore interno di}} \\
      \vspace*{0.2cm}
      {\large {\bf CMS per la fase ad alta luminosità di LHC.}} \\
      \vspace*{0.2cm}
   % {\large {\bf \bm{$H\rightarrow WW\rightarrow2l+2\nu$} con \bm{$\sqrt{s}$}=8 TeV}} \\ 
  %  \vspace*{0.2cm}
   % {\large {\bf e confronto con dati CMS.}} \\
    \vspace*{1cm}
    {\large {\bf Development of an innovative powering }} \\
    \vspace*{0.2cm}
    {\large {\bf scheme for CMS inner tracker for High}} \\
    \vspace*{0.2cm}
    {\large {\bf Luminosity phase of LHC.}}\\
   \vspace*{0.2cm}
    %{\large {\bf and comparison with CMS data.}} \\
 % \vspace*{0.2cm}
 %   {\large {\bf with CMS data.}} \\
        %\vspace*{0.5cm}
    \vspace*{2cm}
    \begin{tabular}{l l}
      {\large \bf{Relatore}}: & {\large {\bf }}\\
      {\large Dott. Giacomo Sguazzoni} & {\large {\bf }}\\
      \\
      \\
      {\large \bf{Correlatore}}: & {\large{\bf }}\\
      {\large Prof. Raffaello D'Alessandro} & {\large {\bf }}\\
      \\
      \\
      {\large \bf{Candidato}}: & {\large{\bf }}\\
      {\large Andrea Fiaschi} & {\large {\bf }}\\
\\
\\
{\large \bf{Anno Accademico}}: & {\large 2017/2018{\bf }}\\
      {\large } & {\large {\bf }}\\
    \end{tabular}
    \vfill
  \end{flushleft} 
 
  \end{minipage}
  \end{center}
\end{titlepage}
