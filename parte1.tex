\chapter{L'alimentazione seriale}
%L'alimentazione seriale dei moduli è il sistema scelto per l'alimentazione, in grado di rispettare i requisiti per il nuovo tracciatore a pixel di fase due. 
\section{Introduzione}
%Nel capitolo precedente si sono discussi i requisiti richiesti per il tracciatore di fase due: tolleranza agli alti livelli di radiazione, alta risoluzione, riduzione del materiale, etc...
%Per rispettare queste condizioni si è deciso di sviluppare un sistema di alimentazione innovativo e alternativo agli attuali convertitori DC-DC
%\footnote{
%  L'attuale tracciatore utilizza un sistema di alimentazione in parallelo dei moduli, in cui i convertitori DC-DC generano localmente le tensioni necessarie al funzionamento del chip.
%}.
%Questi, infatti, non resisterebbero ai più alti livelli di radiazione della fase ad alta luminosità, nè sarebbero in grado di operare efficacemente in presenza del campo magnetico previsto.
%Il nuovo sistema di alimentazione prevede una distribuzione seriale e sarà gestito, all'interno del chip, da circuiti dedicati prodotti in tecnologia CMOS a 65 nm.

In un sistema di alimentazione seriale il generatore fornisce tensione e corrente ad una catena di carichi, nel nostro caso i moduli, posti in successione. A differenza del sistema di alimentazione parallelo, in cui tutti i moduli si trovano allo stesso potenziale di riferimento e in cui la corrente totale erogata è la somma di quella che scorre nei singoli rami, nel sistema seriale la corrente totale è la stessa per tutti i moduli ed è la tensione di lavoro a cambiare. Dato che i moduli, e quindi i carichi, sono tutti nominalmente uguali, la differenza di potenziale su ciascun elemento della catena è la stessa, ma il riferimento di potenziale cambia localmente.
Dato che la corrente viene ``riutilizzata'' in ciascun modulo, il generatore deve fornire correnti minori, rispetto a quelle di un sistema di alimentazione parallelo.
Per questo motivo le dissipazioni nei cavi sono ridotte ed è quindi possibile utilizzare cavi più sottili, riducendo il materiale presente all'interno del tracciatore.
%Questo significa che in ogni elemento scorre la medesima corrente, ma l'intervallo di tensione in cui viene a trovarsi è differente per ciascuno di essi. Fino ad ora il metodo di alimentazione utilizzato era quelo parallelo, tutti i moduli si trovano ad un potenziale comune ma su rami separati, questo implica che nel cavo principale scorre una corrente pari alla somma di quella che passa in tutti i rami e dunque sarà necessario l'utilizzo di un cavo con sezione notevolmente maggiore a quello necessario in una alimentazione seriale.
%Ciò dipende dal fatto che nel caso di alimentazione seriale la corrente sia riutilizzata n-volte riducendo così le perdite di potenza nei cavi, rispetto ad una alimentazione in parallelo, questo consente l'impiego di cavi più sottili riducendo così il materiale all'interno del tracciatore. 
%Inoltre un'alimentazione seriale, rispetto ad una in parallelo, permette di ridurre la potenza dissipata a parità del numero di moduli alimentati.

Infatti è possibile calcolare il rapporto tra la potenza assorbita da n elementi in parallelo e quella di n elementi in serie:
\begin{equation}
\mathrm{W_{parallelo} = n \cdot I \cdot V + (I\cdot n)^2 \cdot R},
\end{equation}
\begin{equation}
\mathrm{W_{serie} = n \cdot I \cdot V + I^2 \cdot R},
\end{equation}
\begin{equation}
\mathrm{\frac{W_{parallelo}}{W_{serie}} = \frac{1+ \dfrac{nRI}{V}}{1+\dfrac{RI}{Vn}}},
\end{equation}
dove R è la resistenza dei cavi, I è la corrente di alimentazione per il singolo elemento e V è la caduta di tensione su ciascun elemtno. 

\section{Lo ShuntLDO per l'alimentazione seriale}
\label{sec:SLDOzener}

Nella progettazione del sistema con alimentazione seriale occorre tener conto che il consumo di ciascun ROC, anche se gli elementi della catena sono tutti identici, dipende dallo stato in cui il ROC si trova e dalle operazioni che sta eseguendo. 
Il ROC, infatti, è un carico dinamico e l'alimentazione deve essere in grado di erogare abbastanza corrente per far fronte ai picchi di assorbimento dei singoli elementi della catena.
Inoltre, le variazioni dei consumi sono molto veloci e gli alimentatori remoti, detti anche di \textit{back-end}, non sarebbero in grado di compensarle in modo rapido: questi saranno posizionati lontano dall'esperimento e i cavi, lunghi anche fino a $\sim 60-80\m$, introducono un ulteriore carico induttivo non trascurabile.

Per queste ragioni un aspetto chiave dell'implementazione di uno schema di alimentazione seriale \`e che le fluttuazioni del carico, il ROC nel nostro caso, non siano visibili esternamente ma siano gestite localmente facendo s\`i che il carico effettivo visto dalla catena sia costante. Questo pu\`o essere ottenuto implementando, in un appropriato circuito di {\em shunt}, un cammino alternativo per la corrente in quel momento non utilizzata dal ROC, tale che $\mathrm{I}_\mathrm{TOT}\sim\mathrm{I}_\mathrm{ROC}+\mathrm{I}_\mathrm{shunt}\sim\mathrm{cost}$. Inoltre \`e necessario che, in prossimit\`a del ROC, venga effettuata l'opportuna regolazione PoL per convertire l'alimentazione in corrente fornita alla catena e quindi a ciascuno dei suoi elementi in una alimentazione in tensione stabile fruibile dal ROC. 
\begin{figure}
\centering
\includegraphics[width=0.5\textwidth]{Immagini/zener2}
\caption{Schema di principio di un circuito regolatore con shunt che utilizza un diodo Zener.}
\label{zener}
\end{figure}
Per flessibilit\`a e versatilit\`a \`e auspicabile che questa elettronica addizionale sia integrata nel ROC stesso per le motivazioni espresse nella Sezione~\ref{IntroROCIT}.

Un possibile schema di principio di un tale circuito \`e visibile in~Fig.~\ref{zener} in cui si sfrutta un diodo Zener contropolarizzato per ottenere la stabilizzazione in tensione al carico `load'. Il funzionamento è molto semplice, la corrente in ingresso induce una caduta di tensione sulla resistenza R che, pur tuttavia, non deve impedire di portare il diodo Zener in contropolarizzazione oltre la tensione di soglia. A questo punto sarà lo stesso diodo a mantenere la tensione ai capi del carico costante assorbendo tutta la corrente non necessaria al carico, agendo da shunt. 
Questo schema però presenta problemi operativi nel momento in cui si mettano più regolatori in parallelo come pu\`o essere necessario fare in applicazioni pratiche come, per esempio, nel caso di un ROC dove servono due tensioni generalmente diverse per la parte digitael e la parte analogica ciascuna fornita da un regolatore diverso o nel caso di un modulo dove pi\`u ROC sono alimentati in parallelo condividendo la stessa corrente in ingresso. Il diodo Zener che, per inevitabili tolleranze di fabbricazione, si trovi all'accensione ad entrare in conduzione prima degli altri assorbir\`a tutta la corrente disponibile con effetti imprevedibili e potenzialmente distruttivi. 

Per ovviare a queste criticità è stato sviluppato un regolatore \textit{Low Drop Out}\footnote{Un regolatore è definito Low Drop Out quando \`e minima la differenza tra la tensione in ingresso e quella in uscita a valle della regolazione; in particolare$\mathrm{V_{dropout}=V_{in}-V_{out}\sim 0.1-0.2 V}$.} accoppiato ad un circuito di shunt capace di generare localmente una tensione fissa e di adattarsi dinamicamente all'assorbimento di corrente, denominato {\em ShuntLDO}, con caratteristiche elettriche tali da superare le problematiche a cui si \`e sopra accennato.

Il regolatore, che va sotto il nome ShuntLDO, \`e stato inizialmente sviluppato dalla collaborazione Atlas per un suo utilizzo nell'ambito del progetto {\em Insertable B-Layer} (IBL)~\cite{IBL}. Infatti la famiglia di ROC disegnati per IBL, FE-I3~\cite{ROCFEI3} e FE-I4~\cite{ROCFEI4}, implementano entrambi circuiti ShuntLDO dimensionati per correnti totali di $\sim 0.5\A$. Pur tuttavia IBL non implementa uno schema di alimentazione seriale perch\'e ad esso fu preferito uno schema di alimentazione pi\`u tradizionale.

Il disegno dello ShuntLDO \`e stato poi tradotto in tecnologia CMOS a $65\nm$ dalla collaborazione RD53 per la sfida di HL-LHC: inizialmente in una versione a $\sim 0.5\A$, necessaria per valutare il dispositivo in $65\nm$, per poi arrivare alle attuali versioni dimensionate per corrente massime di $\sim 2\A$.

%Il punto chiave per una alimentazione seriale è che la corrente, che viene fatta scorrere nella catena, sia maggiore o uguale a quella necessaria per alimentare l'elemento con il consumo più alto. 
%Anche nel caso in cui tutti gli elementi della catena siano identici non è vero che avranno pari consumi, gli stessi dipendono dallo stato in cui si trova il chip e le operazioni che esegue. 
%Il chip è dunque un carico dinamico e in qualsiasi momento l'alimentazione deve essere in grado di erogare abbastanza corrente per far fronte ai picchi di assorbimento dei singoli elementi della catena. 
%Dal momento che queste variazioni sono molto veloci l'alimentatore di back-end non può essere in grado di compensarle in modo rapido\footnote{Gli alimentatori sono posizionati lontano dall'esperimentpo e dunque i cavi introducono un ritardo (proporzionale alla lunghezza degli stessi), non è quindi possibile pensare di alimentare la catena in tensione. L'alimentazione dovrà essere in corrente, a livello di chip sarà poi implementato un meccanismo di bilanciamento, che gestisca le variazioni di corrente assorbita.}. Questo è un punto critico che richiede un attento studio. 
%La soluzione al problema consiste nell'implementazione di un regolatore con shunt all'interno del chip, che generi una tensione fissa e si adatti dinamicamente all'assorbimento in corrente. 
%Per quanto detto è necessario che le fluttuazioni di carico date dal chip non siano visibili esternamente, dovranno perciò essere gestite localmente. Il circuito all'interno di ciascun chip incaricato di svolgere questo compito è il regolatore con ShuntLDO.
%\section{Alimentazione seriale con RD53A}
%Partendo dall'idea di utilizzare un'alimentazione seriale, la prima idea che si può avere, è quella di porre singoli elementi in serie, ciascuno dotati di un regolatore, questa configurazione ha però una criticità molto importante: il fallimento di un singolo elemento rende inutilizzabile l'intera catena, inquanto interrompe il flusso di corrente in tutto il ramo. 
%Va perciò implementato un metodo per ovviare a questo problema e anche per mitigare le fluttuazioni di tensione causate dai vari elementi che influiscono significativamente  sulla tensione generata localmente. Il circuito che si occupa di risolvere queste problematiche è un regolatore di tensione ShuntLDO (Shunt Low Drop Out). 
%lo schema consiste in due loop di controllo accoppiati, il primo fa si che il transistor M4 faccia passare tutta la corrente che non è consumata dal carico attivo. Mentre un voltage regulator (A1+M1) assicura che il carico attivo (nel nostro caso RD53A) sia alimentato con una tensione costante, indipendentemente dalla corrente consumata e indipendentemente dalla correte che viene immessa nella catena seriale.
% 
%\section{S$_{unt}$ L$_{ow}$ D$_{rop}$ O$_{ut}$}
%\section{ShuntLDO}


%L'altro punto di interesse nel grafico è quello definito da massima tensione e massima corrente a cui è possibile alimentare il circuito. 
%Una volta sicuri che non sarà mai utilizzata una corrente maggiore del massimo per alimentare il dispositivo, si ha la sicurezza che nemmeno la tensione massima sarà raggiunta.

\subsection{Il principio di funzionamento dello ShuntLDO}

Per descrivere il suo funzionamento partiamo da uno schema estremamente semplificato che permetta di comprendere l'idea di fondo, facendo riferimento alla Fig.~\ref{SLDOprova}. 
Come anticipato il circuito di ShuntLDO è alimentato dalla corrente $\mathrm{I_{in}}$, attraverso il nodo $\mathrm{V_{in}}$.  Della corrente in ingresso una frazione $\mathrm{I_\mathrm{ref}}$ scorrerà nel ramo di R3. Questa sarà la corrente di riferimento all'interno dello shunt, vedremo in seguito perché. 
La parte che agisce come regolatore di tensione è il mosfet M1 tramite l'amplificatore A1 che equalizza la tensione sul carico al valore di riferimento $\mathrm{V_{ref}}$ aprendo o chiudendo il gate di M1 e, dunque, facendo scorrere più o meno corrente. 
In assenza della parte di shunt ci sarebbe, quindi, un aumento di $\mathrm{I_{ref}}$ e, conseguentemente, di $\mathrm{V_{in}}$. 
Per evitare ciò è stato implementato un ramo in parallelo al carico in cui la corrente che scorre è regolata dal mosfet M4 tramite l'amplificatore A3 che, confrontando tramite un opportuno campionamento la corrente nel ramo di M1 e 1000 volte quella nel ramo di R3, ha lo scopo di mantenere la corrente in ingresso a M1 costante. Il complesso costituito da M4 e A3 è la parte attiva dello shunt: A3 fa sì che la corrente nel ramo con M1 sia 1000 volte quella di riferimento che scorre in R3, poiché modifica $\mathrm{I_{shunt}}$ attraverso la variazione della tensione di gate del mosfet M4. 
La corrente che scorre in M1 è dunque tenuta costante anche quando si hanno variazioni dinamiche del carico.
\begin{figure}[!htbp]
\centering
\includegraphics[width=0.75\textwidth]{Immagini/SLDObase}
\caption{Schema di principio dello ShuntLDO.}
\label{SLDOprova}
\end{figure}
 
La Fig.~\ref{SLDOprinciple} mostra graficamente il funzionamento dello ShuntLDO in cui la corrente totale circolante nella catena seriale si divide, in funzione del tempo, in quella indirizzata nello shunt e in quella richiesta dal carico, nelle sue componenti analogica e digitale; \`e esemplificata anche la situazione da evitare, ovvero la richiesta, per tempi prolungati, di una corrente maggiore rispetto a quella disponibile nella catena.
\begin{figure}[!htbp]
\centering
\includegraphics[scale=.5]{Immagini/ShuntRegulatorPrinciple}
\caption{Andamento qualitativo della corrente assorbita da un ROC in funzione del tempo. Mentre la componente analogica \`e sostanzialmente costante, la componente digitale subisce variazioni, anche repentine. Sono da evitare le situazioni in cui il carico richieda, per lunghi perdiodi di tempo, una corrente in eccesso rispetto a quella circolante nella catena, rappresentata in giallo, mentre picchi veloci possono essere gestiti con opportuno filtraggio.}
\label{SLDOprinciple}
\end{figure}

\subsection{Lo ShuntLDO da $0.5\A$ in 65nm}

 
\begin{figure}[!htbp]
\centering
\includegraphics[scale=.3]{Immagini/SLDO5A}
\caption{Circuito semplificato dello ShuntLDO 0.5 A.}
\label{SLDO5A}
\end{figure}

In Fig.~\ref{SLDO5A} \`e mostrato lo schema elettrico del prototipo di ShuntLDO in versione a $65\nm$ dimensionato per $0.5\A$ di corrente massima.
Rispetto a quanto visto in precedenza si hanno alcune differenze, benché l'idea generale di funzionamento rimanga la stessa. 
In questo schema il carico si trova tra i nodi $\mathrm{V_{out}}$ e $\mathrm{I_{out}}$. Grazie al partitore costituito da R1 e R2, con R1=R2, e all'amplificatore A1, $\mathrm{V_{out}}$ risulta essere il doppio della tensione di riferimento $\mathrm{V_{ref}}$.
%\footnote{$\mathrm{V_{ref}}$ non ha niente a che vedere con $\mathrm{I_{ref}}$.}
L'amplificatore A3, invece, confronta la corrente $\mathrm{I_{ref}}$ che scorre nel ramo di R3 (dato che normalmente i nodi $\mathrm{I_{in}}$ e $\mathrm{R_{in}}$ sono connessi) e quella che scorre in M2 e regola il mosfet M4 di conseguenza. 
La coppia di mosfet M1-M2 costituisce un \textit{current-mirror} configurato in modo tale che il rapporto k tra le due correnti sia uguale a 1000 grazie ad una appropriato dimensionamento dei due dispositivi\footnote{
  In un \textit{current-mirror} il rapporto k è uguale a $\dfrac{W_1L_2}{W_2L_1}$ dove $W_1$ e $L_1$ e $W_2$ e $L_2$ sono, rispettivamente, profondità e lunghezza dei due mosfet.
}. 
Grazie al \textit{current-mirror}, quindi, la corrente che scorre in M1 è 1000 volte $\mathrm{I_{ref}}$ e il sistema costituito dall'amplificatore A2 e il mosfet M3 ha lo scopo di migliorare la precisione di questo current mirror.
Questa scelta progettuale definisce il modo in cui lo ShuntLDO è visto esternamente. Infatti la corrente $\mathrm{I_{in}}$ che scorre nello ShuntLDO pu\`o essere espressa come
\begin{equation}
\mathrm{I_{in} \sim k I_{ref} \sim k \frac{V_{in} - V_{thM6}}{R_3}}
\label{eq:IV05amp}
\end{equation}
dove $\mathrm{V_{thM6}}$ \`e la tensione di soglia del mosfet M6. Quindi il generatore connesso al nodo $\mathrm{V_{in}}$ vede un carico resistivo pari a $\frac{R3}{k}$ in serie con una piccola tensione di offset. Il valore di R3 è, quindi, un importante parametro, la cui scelta determina la tensione nel punto di lavoro per una data corrente e, pi\`u in generale, la pendenza di $\mathrm{V_{in}}$ in funzione di $\mathrm{I_{in}}$ nella zona di regolazione, come visibile in Fig.~\ref{fig:IVSLDO} che mostra il caratteristico grafico corrente-tensione dello ShuntLDO che approfondiremo in seguito.
Il valore di R3 \`e configurabile tramite il terminale indicato con $\mathrm{R_{ext}}$ dove è possibile collegare una resistenza esterna per alterare il valore del resistore integrato nello ShuntLDO.
\begin{figure}[!htbp]
\centering
\includegraphics[width=0.99\textwidth]{Immagini/CaratteristicaSLDO.png}
\caption{Caratteristica corrente-tensione per uno ShuntLDO che mostra $\mathrm{V_{in}}$ (curva blu) e $\mathrm{V_{out}}$ (curva rossa) in funzione di $\mathrm{I_{in}}$; la regolazione di  $\mathrm{V_{out}}$ al valore di $\sim 1.15\V$ inizia per $\mathrm{I_{in}}\sim0.4\A$. Al di sopra di questo valore $\mathrm{V_{in}}$ ha una pendenza pari a $\sim\frac{R3}{k}$.}
\label{fig:IVSLDO}
\end{figure}

Per una corretta funzione di regolazione lo ShuntLDO da $0.5\A$ necessita di una tensione in ingresso minima di circa $1.4-1.5\V$ a fronte di un $\mathrm{V_{out}}$ nominale pari a $\sim 1.2\V$ che \`e, entro qualche decina di mV, il valore di tensione necessario sia per la componente analogica che per la componente digitale del futuro ROC per l'IT. Il rapporto tra la differenza tra tensione di ingresso e tensione di regolazione, pari a $\sim 200\mV$, e la tensione in ingresso rappresenta una stima della potenza minima dissipata dallo ShuntLDO pari quindi al $\sim 15-20\%$. A parit\`a di $\mathrm{I_{in}}$ e di $\mathrm{V_{out}}$, la scelta della resistenza R3 definisce $\mathrm{V_{in}}$ e quindi il punto di lavoro dello ShuntLDO in termini di potenza in eccesso che \`e necessario fornire al regolatore.

% Valori di resistenza maggiori consentiranno, quindi, di operare con correnti minori, con il vantaggio di consumare minor potenza\footnote{
%   La corrente in ingresso è fissata e quella non utilizzata viene dissipata sullo shunt, che diventa conseguentemente molto caldo.
% },
% ma con lo svantaggio di aver minor spazio per le fluttuazioni del carico. 
% Analogamente resistenze minori avranno l'effetto contrario: tensioni minori con correnti maggiori.

\subsection{Lo ShuntLDO da $2\A$ in 65nm}

La successiva versione del prototipo di ShuntLDO a 65nm, progettata per correnti nominali di $2\A$, implementa un'ulteriore parametro di configurazione della caratteristica IV per una maggiore flessibilit\`a di utilizzo e configurazione. In particolare \`e stata aggiunta una parte circuitale che permette di calibrare la tensione di offset che definisce la corrente di riferimento nella resistenza R3, che, nella versione precedente, era semplicemente pari a $\mathrm{V_{thM6}}$. Nella Fig.~\ref{SLDO2A}, che mostra lo schema elettrico dello ShuntLDO da $2\A$, \`e visibile il complesso aggiunto per questo scopo costituito dal mosfet M7 controllato dall'amplificatore A4.
\begin{figure}
\centering
\includegraphics[width=\linewidth]{Immagini/SLDO2A}
\caption{Circuito semplificato dello ShuntLDO 2A.}
\label{SLDO2A}
\end{figure}
%Per quanto visto fin ora non c'è traccia della possibilità di avere un offset.
%In questo tipo di schematico non è prevista la possibilità di avere una $\mathrm{V_{offset}}$.
%Questa caratteristica è stata introdotta nel prototipo a $2A$ riportato in figura \ref{SLDO2A}.
%Facendo riferimento a questa, si può vedere che, rispetto alla versione a $0.5A$, è presente il mosfet M7 sul ramo di R3, il cui gate è controllato da A4.
Grazie alla presenza di questo ulteriore stadio, la corrente di riferimento $\mathrm{I_{ref}}$ diventa, quindi
\begin{equation}
\mathrm{I_{ref} = \frac{V_{in} - V_{ofs}}{R_3}}
\end{equation}
dove $\mathrm{V_{ofs}}$ \`e la tensione di offset definita tramite l'omonimo terminale da cui segue una relazione tra $\mathrm{V_{in}}$ e $\mathrm{I_{in}}$ leggermente modificata rispetto alla Eq.~\ref{eq:IV05amp}: 
\begin{equation}
\mathrm{I_{in} \sim k I_{ref} \sim k \frac{V_{in} - V_{ofs}}{R_3}}.
\end{equation}
La tipologia circuitale dello ShuntLDO a $2\A$ \`e, per il resto, invariata rispetto all'omologo a $0.5\A$. La pi\`u ampia tolleranza in corrente \`e ottenuta grazie al dimensionamento dei mosfet M1 e M4 e ad altri dettagli di implementazione del circuito in tecnologia CMOS.

\subsection{Lo ShuntLDO: il circuito equivalente}

Da quanto descritto nelle sezioni precedenti si pu\`o definire regione operativa o di regolazione dello ShuntLDO quell'ambito in cui la corrente $\mathrm{I_{in}}$ \`e tale per cui $\mathrm{V_{in}\gtrsim 2\times V_{ref}}+{\cal O}\mathrm{(200\mV)}$, e quindi $\mathrm{V_{out}\sim 2\times V_{ref}}$, e, al contempo, $\mathrm{V_{in}\lesssim V_{in}^{MAX}}$ dove $\mathrm{V_{in}^{MAX}\sim2.0\V}$, valore limite per un circuito basato su tecnologia CMOS a $65\nm$. In questa regione il comportamento dello ShuntLDO rispetto ai terminali di ingresso \`e schematizzabile come una resistenza efficace, $\mathrm{R_{eff}}$, definita da $\frac{R3}{k}$ in serie ad un generatore di tensione, $\mathrm{V_{offset}}$. Le variazioni di corrente utilizzata dal reale carico attivo, il ROC nel nostro caso, non sono quindi visibili esternamente al regolatore
%Un importante caratteristica di questo particolare circuito è che esternamente è visto come una resistenza efficace $\mathrm{R_{eff}}$, in serie ad un offset di tensione $\mathrm{V_{offset}}$, mentre il carico attivo, nel nostro caso il chip RD53A, non è visibile e dunque non lo sono nemmeno le sue le sue variazioni. 
Il comportamento resistivo permette l'utilizzo di più ShuntLDO in parallelo fra di loro, con la corrente che si suddivide in base alla resistenza efficace di ciascuno. La resistenza equivalente risultante \`e quella del parallelo delle resistenze efficaci dei singoli dispositivi superando cos\`i le problematiche descritte nella Sezione~\ref{sec:SLDOzener} che affliggono un tradizionale circuito shunt basato su diodo Zener.

%Inoltre, utilizzando resistenze esterne, è possibile scegliere il valore di $\mathrm{R_{eff}}$ e, conseguentemente, la quantità di corrente che scorre in ciascun elemento posto in %parallelo.

Lo ShuntLDO pu\`o essere implementato direttamente sul ROC stesso e questo ne rappresenta uno degli innegabili vantaggi sia in termini di miniaturizzazione, sia in termini di prossimit\`a al carico reale della regolazione PoL. Come menzionato, gi\`a i ROC per Atlas IBL (FE-I3 e FE-I4) ospitavano due ShuntLDO, uno per la tensione analogica e uno per la tensione digitale. La collaborazione RD53 ha recentemente realizzato il ROC RD53A\cite{RD53A}, il primo prototipo del ROC per HL-LHC, destinato ad essere utilizzato per ampi studi di R\&D. Anche RD53A implementa due ShuntLDO destinati all'alimentazione analogica e all'alimentazione digitale. In condizioni normali questi ShuntLDO lavoreranno in parallelo condividendo la corrente in ingresso. La scelta dei rispettivi valori di $\mathrm{R_{eff}}$ definisce la suddivisione di corrente nei due domini. A differenza dei ROC della passata generazione, in cui il consumo di corrente della parte digitale era marginale rispetto alla parte analogica, in RD53A la componente analogica e quella digitale avranno consumi confrontabili. Per questo motivo i due ShuntLDO sono configurati in modo analogo.

La figura~\ref{VVC} mostra l'andamento della tensione $\mathrm{V_{in}}$ in funzione della corrente totale $\mathrm{I_{in}}$ fornita al ROC assumendo la caratteristica ideale corrente-tensione sopra descritta. Sono, inoltre, evidenziati i valori operativi massimi e la tensione minima di lavoro.
\begin{figure}[!htbp]
\centering
\includegraphics[scale=.3]{Immagini/VoltageVsCurrent}
\caption{Andamento della tensione in funzione della corrente per il chip. Le zone tratteggiate sono oltre i valori operativi massimi (4.0 A per la corrente e 2.0 V per la tensione), la linea orizzontale a 1.4 V è la tensione minima di lavoro, la pendenza è la resistenza efficace. Combinazioni differenti di resistenza e offset consentono di spostare il punto di lavoro.}
\label{VVC}
\end{figure}
In particolare vengono illustrati i criteri di scelta per i valori di $\mathrm{R_{eff}}$ e $\mathrm{V_{offset}}$: da un lato \`e necessario avere una corrispondenza fra la massima corrente erogabile ($4.0\A$ per il parallelo dato che il singolo ShuntLDO tollera una corrente massima di $\sim2\A$) e la massima tensione ($2.0\V$), dall'altra si dovrà far s\`i che la corrente corrispondente alla tensione minima di lavoro ($1.4\V$) sia sufficiente per il funzionamento del ROC tenendo anche conto di un certo margine necessario per sopperire a eventuali fluttuazioni nei consumi. Nella figura sono visibili due possibili combinazioni $\mathrm{R_{eff}}$ e $\mathrm{V_{offset}}$ rappresentate come due diverse caratteristiche IV.

% Una volta scelti i valori di $\mathrm{R_{eff}}$ e $\mathrm{V_{offset}}$, il consumo in potenza è completamente definito da:
% %La scelta di $\mathrm{R_{eff}}$ e $\mathrm{V_{offset}}$ definisce il consumo in potenza:
% \begin{equation}
% \mathrm{W=I_{in}^2R_{eff}+I_{in}V_{offset}}
% \end{equation}
% %una $R_{eff}$ minore consente di avere minor aumento di potenza all'aumentare della corrente. 
% %e dunque per un corretto funzionamento del chip è possibile utilizzare correnti minori, ovvero lasciando meno margine per le fluttuazioni, proprio perchè con $r_{eff}$ minore le fluttuazioni di corrente
% Accenniamo qui al fatto che la possibilità di modificare $\mathrm{R_{eff}}$ e $\mathrm{V_{offset}}$ permette di progettare diverse configurazioni di consumo in relazione alle necessità.
% Ad esempio si potrebbe avere un \textit{high power mode}, in cui la corrente alla tensione minima di lavoro sia sufficiente a garantire il funzionamento dei chip accesi e funzionanti alle massime prestazioni, come potrebbe essere il caso della linea rossa in figura \ref{VVC}, e un \textit{low power mode} in cui la stessa sia minore poiché non si richiede il funzionamento ``completo'' dei chip, come ad esempio rappresentato dalla linea blu in Figura \ref{VVC}.


% %Con questo tipo di configurazione la corrente erogata dagli alimentatori di back-end deve essere maggiore o uguale a quella necessaria all'elemento con consumo più alto.

% %La posibilità di modificare $\mathrm{R_{eff}}$ e $\mathrm{V_{offset}}$ apre alla possibilità di progettare un modo per avere diverse configurazioni, una di low-power mode e una high power mode. %Al fine di ridurre il consumo di potenza, la resistenza effettiva dello SLDO può avere un offset modificabile. 
% %(Low-power mode configuration, che cosa posso dire...).




\subsection{Schema di catena di alimentazione con ShuntLDO}

\begin{figure}
\centering
\includegraphics[scale=.3]{Immagini/MultiChipModules}
\caption{Schema di una catena seriale con due moduli da quattro chip. Il ramo con $\mathrm{R_{eff}}$ e $\mathrm{V_{offset}}$ in colore rosso rappresentano un ROC malfunzionante, non pi\`u attraversato da corrente, come discusso nel testo.}
\label{MCM}
\end{figure}

Il disegno dell'IT prevede moduli 1x2 e 2x2 organizzati in catene di alimentazione seriale composte da 8-12 moduli. I due o quattro ROC di ciascun modulo sono alimentati in parallelo. Lo schema equivalente di due moduli da quattro ROC posti in serie \`e mostrato in Fig.~\ref{MCM} dove $\mathrm{R_{eff}}$ e $\mathrm{V_{offset}}$ rappresentano, a loro volta, il circuito equivalente al parallelo dei due ShuntLDO che sovrintendono, rispettivamente, alla alimentazione analogica e alla alimentazione digitale.

Nella fase progettuale era stata valutata anche la possibilit\`a di implementare l'alimentazione seriale concatenando in serie 8-12 ROC singoli; la catena sarebbe cio\`e costituita da un certo numero di moduli in serie con i ROC di ciascun modulo concatenati in serie anch'essi. Un tale approccio \`e stato scartato perch\'e il sensore si troverebbe ad operare con i pixel polarizzati ad una tensione differente a seconda del ROC a cui sono connessi con effetti che non sono stati valutati a fondo e che, molto probabilmente, richiederebbero uno specifico disegno delle strutture del sensore.

Andiamo adesso a studiare l'andamento dei consumi dovuti a più moduli in serie e come questi cambiano nel caso di un malfunzionamento di uno di questi (per esempio al livello del regolatore) in seguito al quale si azzera la corrente che scorre nel ramo ad esso relativo.
%Per meglio comprendere i risvolti dati dall'utilizzo di un'alimentazione seriale con SLDO è interessante fare un esempio dei consumi di questo tipo di alimentazione. 
A titolo di esempio, quindi, prendiamo la situazione in cui si ha una catena seriale costituita da 8 moduli da 4 ROC ciascuno; 
% questo per far si che nel caso un regolatore si guasti , la corrente possa essere assorbita da un altro regolatore all'interno del modulo\footnote{Come già detto i regolatori devono essere in grado d lavorare in parallelo, generare differenti tensioni dalle correnti in ingresso e tenere costante il consumo di corrente tramite lo shunt.}. 
assumiamo che il punto di lavoro scelto corrisponda a $\mathrm{R_{eff}=0.3\Ohm}$ $\mathrm{V_{offset}=0.8}\V$ per il parallelo tra lo ShuntLDO `analogico' e lo ShuntLDO `digitale' dato che la resistenza equivalente di un singolo ShuntLDO \`e tipicamente $0.600\Ohm$.
%Prendiamo come modello la situazione in cui si ha un serie di 8 moduli con 4 chip ciascuno, assumiamo che $\mathrm{V_{offset}=0.8} \V$ e $\mathrm{R_{eff}=0.3}$ $\Omega$ per ciascun chip\footnote{$\mathrm{R_{eff}}$ del singolo SLDO è circa 0.600 $\Omega$, nel chip sono presenti due SLDO in parallelo uno per l'alimentazione della parte digitale ed uno per quella analogica. La $\mathrm{R_{eff}}$ con cui viene visto il chip è dunque la metà, 0.300 $\Omega$.}. 
Con una corrente di lavoro totale richiesta di $\sim\mathrm{I^{ROC}_{in}=2.0}\A$ per ROC, equamente divisa tra parte analogica e parte digitale, la corrente totale da fornire all'intero modulo \`e $\sim \mathrm{I_{in}^{modulo}=8.0}\A$ corrispondente a $\mathrm{V_{modulo}=1.4} \V$.

Rispetto a questo valore nominale la corrente circolante nella catena deve essere incrementata per avere un opportuno margine per gestire picchi di carico da parte del ROC a valle del regolatore. Come si vedr\`a pi\`u avanti, per esempio in Fig.~\ref{fig:LoadTransient}, per avere stabilit\`a della regolazione \`e ragionevole avere un $\sim20-25\%$ di corrente in pi\`u rispetto al valore nominale.
% Per avere un po' di margine incrementiamo la corrente di un 20$\%$, la scelta di dare un margine del 20$\%$, piuttosto che del 40$\%$ è una scelta giustificata qualitativamente dall'ampiezza dei picchi di tensione generati dallo SLDO in concomitanza di un transiente, in figura \ref{LoadTransient} è riportato il valore di picco dell'undershoot/overshoot in funzione del margine di corrente dato su 1 A di alimentazione.
% \begin{figure}
% \centering
% \includegraphics[width=0.75\textwidth]{Immagini/LoadTransientDominik}
% \caption{Il grafico riporta il valore di picco dell'undershoot/overshoot in funzione del margine di corrente dato su 1 A di alimentazione nel caso in cui  i consumi di parte analogica e digitale sono entrambi $\sim$0.5 A, 100$\%$ significa che la corrente fornita al chip è 1 A+1 A.}
% \label{fig:LoadTransient}
% \end{figure}
Conseguentemente $\mathrm{I^{ROC}_{in}=2.4}\A$, $\mathrm{I_{in}^{modulo}=9.6}$ A, $\mathrm{V_{modulo}\sim1.5} \V$ e, quindi, la caduta di tensione su tutta la catena, formata da 8 moduli, sarà $\sim 12\V$.
%Con una corrente di $\mathrm{I_{in}=2.0}$ A per chip\footnote{Che vale a dire 1 A per ciascuno dei due SLDO presenti nel chip.} ($\mathrm{I=8.0}$ A per modulo) si avrà un $\mathrm{V_{modulo}=1.4} \V$, per avere un po' di margine incrementiamo la corrente di un 20$\%$, la corrente che arriva nei moduli sarà I = 9.6 A. 
%In questa situazione $\mathrm{V_{modulo}=1.52}$ V e dunque la caduta di tensione su tutta la catena formata da 8 moduli sarà pari a $12.16 \V$. Questa non è la caduta di tensione totale, va tenuto conto anche della resistenza dovuta ai cavi, in questo esempio assumiamo valga 2 $\Omega$. 
%Tenendo conto anche della resistenza dovuta ai cavi, che assumiamo valga $2 \Ohm$,
La tensione fornita generatore remoto sar\`a $\sim 14-15\V$ perch\`e si dovr\`a tener conto anche della dissipazione sui cavi che si stima compresa tra il $15$ e il $20\%$, comunque ben inferiore ai valori tipici di uno schema di alimentazione parallelo.
%Il generatore sarà infine $\mathrm{V \sim I_{in}^{modulo} \cdot 2 \Ohm +12 \V \sim 31 \V}$.
 
% Dai conti fatti risulta che circa il $60 \%$ della potenza è dissipata sui cavi\footnote{
%  Nonostante il valore elevato in assoluto, bisogna tenere presente che è in realtà un miglioramento rispetto alla situazione attuale, nella quale l'alimentazione è in parallelo.
%}. % sarebbe carino sapere la percentuale di consumo attuale
%La tensione di uscita del generatore è dunque $\mathrm{V=I \cdot 2 \Omega +12.6 \V=31.8 \V}$\footnote{Circa il $60 \%$ della potenza è dissipata sui cavi, questo fatto potrebbe sembrare un pessimo traguardo, in realtà è un miglioramento rispetto alla situazione attuale, nella quale l'alimentazione è in parallelo.}.% sarebbe carino sapere la percentuale di consumo attuale

% Il generatore posto a monte della catena è limitato in corrente e, per fare sì che la potenza erogabile sia maggiore di quella necessaria alla catena, si pone il valore limite della tensione leggermente superiore a quello fin qui calcolato, diciamo ad esempio 34 V.
% %In questo sistema il generatore a monte della catena sarà limitato in corrente, mentre il limite per la tensione sarà posto leggermente maggiore a quello minimo, ad esempio 34 V. 
% %In questo modo la potenza che il generatore può erogare è maggiore di quella necessaria per la catena. 
% La potenza massima che il generatore potrà erogare è, quindi, di $34 \V \cdot 9.6$ A = 326.4 W, per cui, sottraendo la potenza dissipata sui cavi e dividendo per il numero di moduli, si ottiene una potenza per modulo di 17.76 W
% \footnote{
%   Questo a fronte di una potenza assorbita, nelle normali condizioni di lavoro, di
%   \begin{equation*}
%     \mathrm{W_{modulo} = 4 \cdot W_{chip} = I_{modulo} \cdot V_{modulo} = 14.6 W}
%   \end{equation*}
%   per modulo.
% }.
% %Vedremo infatti, come questo sia necessario utile nel caso in cui alcuni chip smettano di funzionare. 
% %La potenza massima erogabile dal generatore sarà $34 \V \cdot 9.6$ $A = 326.4$ W, sottraendo la potenza dissipata sui cavi e dividendo per il numero di moduli si ottiene una potenza per modulo di $17.76$ W. Questo a fronte  di una potenza assorbita, nelle normali condizioni di lavoro, di $\mathrm{W = 4 \cdot W_{chip} = I \cdot V_{modulo} = 14.6}$ W per modulo.

Una accurata valutazione dei malfunzionamenti della catena seriale non \`e ancora stata fatta nell'ambito del progetto IT. Questa infatti non pu\`o prescindere dalla realizzazione di catene prototipali estese, attivit\`a prevista nei prossimi mesi. Ipotizziamo tuttavia che, a seguito di un danneggiamento, gli ShuntLDO di un ROC non siano pi\`u attraversati da corrente come mostrato in Fig.~\ref{MCM} dal ramo con $\mathrm{R_{eff}}$ e $\mathrm{V_{offset}}$ rappresentati in colore rosso.
% Vediamo a questo punto cosa accade nel caso in cui uno dei chip in uno dei moduli sia fuori uso
% \footnote{
%   Dal momento che nel chip ci sono due SLDO, in realtà lo scenario più probabile è che solo uno dei due sia danneggiato.
% },
%Vediamo a questo punto cosa accade nel caso in cui un chip in uno dei moduli sia fuori uso\footnote{Dal momento che nel chip ci sono due SLDO, in realtà lo scenario più probabile è che solo uno dei due sia danneggiato.}.
Ciascuno dei tre ROC dello stesso modulo rimanenti dovrà quindi farsi carico della corrente che non attraversa pi\`u il modulo danneggiato per una corrente totale per ROC pari a $\mathrm{I_{in}^{modulo}/3 \sim 3.2 A}$, valore corrispondente a un terzo di corrente in più rispetto ai $2.4\A$ nominali.
%Dato che l'alimentazione è in corrente, ciascuno dei tre chip rimanenti dovrà assorbire un terzo di corrente in più, dunque la corrente per ciascun chip sarà $\mathrm{I = 9.6A / 3 = 3.2 A}$. 
Di conseguenza cambier\`a anche la caduta di tensione sul modulo in question che sarà 
\begin{equation*}
  \mathrm{V_{modulo} = 3.2\A \cdot 0.3 \Omega + 0.8\V = 1.8 \V},
\end{equation*}
e la potenza dissipata
\begin{equation*}
  \mathrm{W_{modulo} = 3 \cdot W_{chip} = 3 \cdot I \cdot V_{modulo} = I_{modulo} \cdot V_{modulo} = 16.9W},
\end{equation*}
con un incremento di $2\W$ per il singolo modulo, corrispondente a circa il $13\%$. Sul totale della catena, questa maggior richiesta di potenza \`e dell'ordine del \% e quindi marginale. Un ulteriore malfunzionamento porterebbe la caduta di tensione sul modulo a $\mathrm{V_{modulo} = 4.8\A \cdot 0.3 \Omega + 0.8\V = 2.2 \V}$, quindi oltre la soglia di tolleranza in tensione della tecnologio CMOS a $65\nm$ con la conseguenza di avere l'imminente rottura anche degli ulteriori due ROC. La situazione \`e in generale assai pi\`u critica per i moduli 1x2 dato che, in caso di malfunzionamenti, il ROC rimanente si troverebbe a gestire il doppio della corrente nominale.

Si spera per\`o che risulti chiaro da questi esempi che la scelta del punto di lavoro (e quindi di  $\mathrm{R_{eff}}$ e $\mathrm{V_{offset}}$) \`e di importanza fondamentale non solo per le condizioni operative standard, ma anche per la gestione dei malfunzionamenti che spostano localmente il punto di lavoro dei regolatori adiacenti. Non \`e detto per\`o che il meccanismo di rottura pi\`u frequente sia quello in cui il ROC e i suoi regolatori non sono pi\`u attraversati da corrente; \`e ipotizzabile che gli scenari pi\`u frequenti di rottura coinvolgano il solo ROC vero e proprio e che quindi gli ShuntLDO ad esso relativi continuino a funzionare. In questo caso l'inconveniente \`e che tutta la potenza viene dissipata sui mosfet M4 dei regolatori. Per questo motivo nell'implementazione reale dello ShuntLDO sul ROC RD53A e versioni successive, la parte di potenza (essenzialmente il ramo costituito da M1 e M4) \`e ottenuta da un certo numero di repliche connesse in parallelo e distribuite sulla superficie del ROC per evitare un potenziale `hot spot' sia in condizioni standard di funzionamento che in caso di malfunzionamenti.

% Per la catena di 8 moduli l'incremento è, invece, di appena $1.7\%$.
% %Questo porta ad una caduta di tensione sul modulo $\mathrm{V_{modulo} = 3.2 A \cdot 0.3 \Omega + 0.8 V = 1.76 \V}$ e una potenza dissipata $\mathrm{W = 3 \cdot W_{chip} = 3 \cdot I \cdot V_{modulo} = 16.9}$ W, con un incremento di 2 W per il singolo modulo, che corrisponde al $13 \%$. Per la catena di 8 moduli l'incremento, invece, è di appena $1.7\%$. 
% Inoltre ci sarà anche un lieve aumento della tensione erogata dal generatore, $\mathrm{V=I \cdot 2 \Omega +14.08 \V=33.28 \V}$, che è comunque al di sotto dei 34 V
% \footnote{
%   Come visto alla tensione massima di  34 V il generatore riesce a distribuire una potenza di $17.76$ W per ciascun modulo.
% }
% impostati.
%Questo causerà anche un lieve aumento della tensione erogata dal generatore, che però rimarrà al di sotto dei 34 V\footnote{Come visto alla tensione massima di  34 V il generatore riesce a distribuire una potenza di $17.76$ W per ciascun modulo.}.
 %mettere un recap con il senso di questi conti
%Lo studio dell'alimentazione seriale parte dunque dalla caratterizzazione dello SLDO. Componente che sarà poi utilizzato all'interno del chip RD53A per la generazione delle tensioni di alimentazione della parte analogica e digitale. .......continuare discorso....
%Aver chiaro come il chip viene visto esternamente e quali siano i suoi consumi è importante per poter progettare al meglio il sistema di alimentazione seriale.
%capitolo
%\section{Prova}



%At module level, the current to voltage conversion should
%be done using more than one regulator, and by connecting all regulators in parallel. In this
%way, should one regulator fail, the current flow can still be assured by the other regulators on
%module. Although these measures assure a very robust design, extra care has to be taken for
%possible worst case failures, in particular for the case of regulator faults which could cause an
%over-voltage

%\subsection{PCB}
%
%Fino ad ora ci siamo limitati alla descrizione del funzionamento dello SLDO.
%Prima di procedere alla presentazione di misure introduciamo brevemente la (\textit{Printed Circuit Board}) di test, nella cui parte centrale è stato collocato e connesso, con wire-bond, lo ShuntLDO.
%La PCB riportata in figura \ref{PCBTestSLDO} è quella relativa al prototipo di ShuntLDO da 2A.
%La PCB è fornita di tutto ciò che è necessario al funzionamento dello SLDO e all'esecuzione dei test basilari: sono presenti connettori molex per l'alimentazione, jumper di configurazione, pin per misurare varie tensioni, etc....
%\begin{figure}
%\centering
%\includegraphics[scale=.3]{Immagini/chipcard}
%\caption{PCB di Test per lo SLDO 2A.}
%\label{PCBTestSLDO}
%\end{figure}


\section{Caratterizzazione ShuntLDO da 0.5A}

Riportiamo in questo paragrafo le prime misure effettuate sul prototipo di ShuntLDO da 0.5A. A differenza della versione a 2A, in questa non è possibile inserire un valore di offset regolabile al $\mathrm{V_{out}}$.
Come primo studio abbiamo caratterizzato lo ShuntLDO, utilizzando una alimentazione in corrente.
Le misure effettuate sono servite ad esaminare l'andamento della tensione di $\mathrm{V_{in}}$, $\mathrm{V_{out}}$ e $\mathrm{V_{ref}}$ in funzione della corrente in ingresso.

\subsection{Misure in assenza di carico}

\begin{figure}
\centering
\includegraphics[scale=.5]{Immagini/provaSLDO5}
\caption{V$_{in}$ (blu), V$_{out}$ (rosso) e V$_{ref}^2$ (giallo) in funzione della corrente di ingresso.}
\label{provaSLDO5}
\end{figure}

In figura \ref{provaSLDO5} è riportato il grafico ottenuto variando la corrente in ingresso, in configurazione senza carico sul $\mathrm{V_{out}}$.
Questa, quindi, è tale per cui tutta la corrente fornita dall'alimentazione viene assorbita dallo shunt.
La parte a sinistra del grafico corrisponde alla situazione in cui il regolatore non è attivo. Andiamo, quindi, a considerare l'andamento asintotico riportando, in particolare, alcuni valori di interesse:

\[
\begin{array}{ccccc}

\toprule
V_{ref} & V_{out} & 2 \cdot V_{ref}- V_{out} & R_{eff} & V_{offset} \\

\midrule

0.516 V & 1.028 V & 8 mV & 2.0 \Omega & 0.40 V \\

\bottomrule
\end{array}
\]

%Come detto, in questa prima misura non è stato applicato nessun carico al $\mathrm{V_{out}}$, dunque tutta la corrente in ingresso scorre nello Shunt, andando a scaldare molto l'oggetto, questa misura ci aiuterà in un confronto successivo con situazioni in cui è presente un carico, sia statico che dinamico. 
Questa misura in assenza di carico al $\mathrm{V_{out}}$, in cui tutta la corrente in ingresso scorre nello shunt scaldandolo significativamente, è di particolare interesse per i confronti con la configurazione in cui è presente un carico, sia esso statico o dinamico.

\subsection{Misure di due ShuntLDO in serie con carico da $\mathrm{4 \Omega}$}

%La successiva misura di test che è stata eseguita con questo prototipo di shunt, prima del passaggio alla versione da 2A, è un serie di due SLDO entrambi con un carico resistivo di $\mathrm{4 \Omega}$. In questo caso i due elementi hanno $\mathrm{V_{ref}}$ diversi $\mathrm{V_{ref1}=0.497}$ V e $\mathrm{V_{ref2}=0.553 V}$. 
%Ciò non rappresenta un problema in quanto il comportamento "esterno " non ne è influenzato.
Sulla PCB è presente un bandgap la cui alimentazione può essere separata da quella dello ShuntLDO. Il compito di questo bandgap è quello di generare la tensione di riferimento $\mathrm{V_{ref}}$, il cui valore può essere regolato con un potenziometro presente sulla PCB.
In figura \ref{SLDO5Serie} possiamo vedere il confronto fra il diverso comportamento del serie di due ShuntLDO nel caso in cui VCC, tensione che alimenta il bandgap, sia esterna e quello in cui sia cortocircuitata con l'ingresso di $\mathrm{I_{in}}$, trovandosi, dunque, a tensione $\mathrm{V_{in}}$.
In questo confronto va tenuto conto che il bandgap presente sulla PCB ha un regime di lavoro compreso tra i 2 V e i 18 V. 
\begin{figure}
\centering
%\subfloat[][Fondo $WW$.]
\includegraphics[width=.85\textwidth]{Immagini/SLDO5Serie1}
%\subfloat[][Fondo $WW$.]
\includegraphics[width=.85\textwidth]{Immagini/SLDO5Serie2}
\caption{In alto i bandgap sono alimentati esternamente, in basso sono alimentati tramite $\mathrm{V_{in}}$.}
\label{SLDO5Serie}
\end{figure}
%Si può notare dai grafici come finto a che la tensione di ingresso non supera circa 1 V il bandgap non riesce a generare il giusto livello di tensione $\mathrm{V_{ref}}$ e ciò ha come conseguenza un $\mathrm{V_{out}}$ non stabile. Si arriva ad una stabilità del $\mathrm{V_{out}}$, solo a tensioni in ingresso più elevate e dunque, con correnti maggiori.
Dai grafici si può notare che se la tensione di ingresso è inferiore ad 1 V, il bandgap non è in grado di generare il giusto livello di tensione $\mathrm{V_{ref}}$, risultando in valori di $\mathrm{V_{out}}$ non stabili.
La stabilità di $\mathrm{V_{out}}$ si può ottenere solo con tensioni in ingresso più elevate e, dunque, con correnti maggiori.

%magari fare tabellino anche qui
\subsection{Comportamento dinamico}

\begin{figure}[!hbt]
\centering
\includegraphics[scale=.5]{Immagini/SLDO5singlepulse}
\caption{Entità degli undershoot in tensione in funzione della corrente assorbita dal mosfet.}
\label{SLDO5singlepulse}
\end{figure}

%Prima di passare alla versione da 2A è stata provata una situazione con carico dinamico, per poi riproporla in modo più approfondito ed esaustivo nelle sezioni successive utilizzando però il prototipo da 2A.
I test con carico dinamico qui riportati sono stati ripresi ed approfonditi con il prototipo a 2A e saranno descritti nelle sezioni successive.
Nella PCB di test è presente, in parallelo all'uscita di $\mathrm{V_{out}}$, un mosfet in serie ad una resistenza, il cui comportamento può essere controllato applicando una tensione dall'esterno sul gate. 
La corrente assorbita dal mosfet, si ricava misurando la caduta di tensione su questa. 

La prima misura di interesse è la sensibilità di $\mathrm{V_{out}}$ alle variazioni veloci di carico.
In figura \ref{SLDO5singlepulse} è riportato l'andamento dell'undershoot in funzione della corrente assorbita dal mosfet.
Le misure sono state effettuate con una alimentazione in corrente $\mathrm{I_{in} = 0.5 A}$, $\mathrm{V_{ref} \sim 0.5 V}$ e carico resistivo su $\mathrm{V_{out}}$ di 4 $\Omega$. 
Conoscere il valore della corrente in ingresso è importante poiché, nel momento in cui il carico dinamico e statico assorbono una corrente maggiore di quella totale in ingresso, il sistema smette di funzionare in maniera corretta.
Gli effetti visibili, nelle situazioni in cui $\mathrm{I_{mosfet} + I_{load} > I_{in}}$, non sono direttamente legati alle prestazioni dello ShuntLDO.Al contrario si può vedere dal grafico in figura \ref{SLDO5singlepulse} come gli undershoot siano inferiori a 10 mV fintantoché $\mathrm{I_{mosfet}}$ rimane sotto gli 0.250 A, valore oltre cui $\mathrm{I_{mosfet} + I_{load} > I_{in}}$
\footnote{
  $\mathrm{I_{load}}$ è la corrente che scorre nel carico resistivo, in questo caso $\mathrm{I_{load} = \dfrac{V_{out}}{R} = 0.250 A}$.
}. 

Lo stesso tipo di test può essere eseguito mettendo due ShuntLDO in serie e verificando, al variare del carico di uno, che l'altro elemento della catena seriale ne sia influenzato o meno.

%ed andando a variare dinamicamente il carico di uno dei due, verificando se esternamente queste variazioni siano visibili, e quindi se influenzino l'altro elemento della catena seriale. 
%Dal momento che l'interesse maggiore è per il prototipo a 2 A questo tipo di misura non è stato riportato nel caso dello SLDO da 0.5A, in quanto lo scopo di questa prima parte è quello di introdurre un certo tipo di approccio e  prendere confidenza con gli argomenti trattati.

\section{ShuntLDO 2A}
%\begin{figure}
%\centering
%\includegraphics[scale=.3]{Immagini/PCB2A}
%\caption{.}
%\label{PCB2A}
%\end{figure}
%Rispetto a quanto visto in precedenza, nella versione da 2 A è possibile gestire anche l'offset attraverso un potenziometro, che va ad agire sulla tensione in uscita generata dal bangap, la stessa che viene utilizzata anche per generare $\mathrm{V_{ref}}$ regolando un secondo potenziometro. 
Come abbiamo già detto, un'importante differenza fra la prima versione, con carico massimo da 0.5 A, e la versione da 2 A è la possibilità di modificare l'offset attraverso un potenziometro. In questo modo si può regolare la tensione in uscita generata dal bandgap, la quale può essere usata come $\mathrm{V_{ref}}$ per un secondo potenziometro.

In questa sezione descriveremo i test effettuati per la caratterizzazione dello ShuntLDO, studiando sia il comportamento con carico statico che dinamico.

\subsection{Comportamento statico}

La caratterizzazione con carico statico è stata ottenuta usando un carico resistivo da 1 $\Omega$ su $\mathrm{V_{out}}$ e ponendo $\mathrm{V_{ref}} = 0.5 \V$.
In questa configurazione il bandgap è alimentato esternamente con una tensione di 5V.
La figura \ref{SLDO2Astatic} mostra i valori misurati di $\mathrm{V_{out}}$, $\mathrm{V_{ref}}$ e $\mathrm{V_{in}}$ al variare della corrente di alimentazione in ingresso.
%La parte di caratterizzazione statica è di nuovo eseguita andando a variare la corrente di alimentazione in ingresso e al contempo misurando $\mathrm{V_{out}}$, $\mathrm{V_{ref}}$ e $\mathrm{V_{in}}$, figura \ref{SLDO2Astatic}.

\begin{figure}
\centering
\includegraphics[scale=.5]{Immagini/SLDO2Astatic}
\caption{Grafico corrente-tensione che riporta gli andamenti di $\mathrm{V_{in}}$, in blu, $\mathrm{V_{ref}\cdot 2}$, in arancione e $\mathrm{V_{out}}$ in rosso.}
\label{SLDO2Astatic}
\end{figure}

Si può notare che $\mathrm{V_{ref}}$ non è costante nella parte con bassa $\mathrm{I_{in}}$: questo comportamento è dovuto al fatto che, nella fase in cui lo ShuntLDO non è attivo, nel ramo di $\mathrm{V_{ref}}$ scorre un po' di corrente che causa una maggiore caduta di tensione del potenziometro e, quindi, una minore tensione all'ingresso di A4 (figura \ref{SLDO2A}).
Infatti, dato che $\mathrm{V_{ref}}$ si trova all'ingresso del comparatore A1, in una situazione di equilibrio viene confrontata con una tensione molto simile.
Il comparatore ha una resistenza molto grossa e la corrente che scorre in questo sarà, quindi, molto piccola.
Nella fase di accensione, invece, all'ingresso di A1 si trova una grossa differenza fra + e - e, quindi, una corrente non trascurabile, causa, come visto, della variazione di $\mathrm{V_{ref}}$.
%Questo andamento è stato ottenuto ponendo
%\footnote{Per selezionare il valore voluto è necessario regolare il potenziometro RP1 che si trova in serie al bandgap sulla PBC.} 
%$\mathrm{V_{ref} = 0.5} V$ e applicando un carico resistivo di 1 $\Omega$ a $\mathrm{V_{out}}$. In questa situazione il bandgap è alimentato esternamente con una tensione di 5 V. 
%Come si vede dal grafico \ref{SLDO2Astatic} $\mathrm{V_{ref}}$ non è costante nella parte iniziale, questo può essere dovuto al fatto che nella fase in cui lo shunt LDO non è attivo si ha uno scorrimento di corrente nel ramo di $\mathrm{V_{ref}}$, ciò causa una maggiore caduta di tensione sul potenziometro e quindi una minore tensione all'ingresso di A4. 
%Idealmente, nel ramo di $\mathrm{V_{ref}}$ dovrebbe scorrere una corrente molto piccola in regime di lavoro\footnote{Questo perché nello SLDO il $\mathrm{V_{ref}}$ è in ingresso al comparatore A1 e viene confrontato con una tensione che sarà circa uguale in una situazione di equilibrio. La corrente che scorre tra ingresso + e - sarà piccola, poiché il comparatore in ingresso ha una grossa resistenza.}, mentre al momento dell'accensione, all'ingresso di A1 si ha una notevole differenza tra + e -, e quindi scorrerà una corrente maggiore, questo è causa di una variazione del $\mathrm{V_{ref}}$. Di seguito riportiamo in tabella i valori che si riferiscono al grafico \ref{SLDO2Astatic}: 

\par \begin{center} !!! SERVE??? COSA CI SI IMPARA? !!! \end{center} \par
\begin{center}
\begin{tabular}{cccccc}
\hline
$\mathrm{V_{ref}}$ & $\mathrm{V_{out}}$ & $\mathrm{2 \cdot V_{ref}- V_{out}}$ & $\mathrm{R_{eff}}$ & $\mathrm{V_{offset}}$ \\
\hline
0.500 V & 0.980 V & 20 mV & 0.880 $\Omega$ & 0.40 V\\
\hline
\end{tabular}
\end{center}
\FloatBarrier

\subsection{Differenze tra GND PCB e GND ShuntLDO}

\begin{figure}[!ht]
\centering
\includegraphics[scale=.3]{Immagini/Ground}
\caption{Differenze tra $\mathrm{GND_{PCB}}$ e $\mathrm{GND_{REG}}$ nella configurazione ShuntLDO, causate dalla corrente di shunt.}
\label{Ground}
\end{figure}
Prima di procedere oltre è interessante fare alcune riflessioni.% per porre attenzione su un aspetto che influenza le misure. 
Tutte le tensioni misurate, come per esempio il $\mathrm{V_{out}}$, sono prese da pin/connettori/piste sulla PCB.
Questo vuol dire che, per esempio, quando viene misurato $\mathrm{V_{out}}$, la tensione letta sull'oscilloscopio o con i multimetri è la differenza di tensione tra il pin di $\mathrm{V_{out}}$ e la terra (GND) della PCB. 
Dal momento che questi punti di misura sono collegati al $\mathrm{V_{out}}$ dello ShuntLDO attraverso \textit{wire bond}, si introduce una resistenza che causa una caduta di tensione. 
L'entità di questa caduta di tensione dipende dalla resistenza dei \textit{wire bond} e dalla corrente, quindi, nel caso di una caratterizzazione con carico statico, si presenta come un offset, mentre nel caso di carico dinamico, varia con la corrente. 
I \textit{wire bond} si presentano come tante resistenze in parallelo, perciò il valore della resistenza equivalente dipende dal numero di queste ultime: maggiore il numero minore il valore della resistenza equivalente e, perciò, minore l'effetto su $\mathrm{V_{out}}$.
%Questo problema è riportato schematicamente in figura \ref{Ground}.

%Ricapitolando, nella configurazione di ShuntLDO una frazione della corrente in ingresso scorre attraverso il transistor di shunt (M4), questa corrente confluisce nella linea di terra del regolatore $\mathrm{GND_{REG}}$ e da qui, attraverso i \textit{wire bond} è collegata alla terra della PCB ($\mathrm{GND_{PCB}}$). 
Facendo riferimento alla figura \ref{Ground}, nella configurazione di ShuntLDO, una frazione della corrente in ingresso scorre attraverso il transistor di shunt (M4), confluisce nella linea di terra del regolatore $\mathrm{GND_{REG}}$ e da qui attraverso i \textit{wire bond} arriva alla terra della PCB ($\mathrm{GND_{PCB}}$). 
La resistenza di questa linea, schematizzata in figura con una resistenza $\mathrm{R_{GND}}$, è la causa della differenza in tensione tra la terra della PCB e dello Shunt. 
Questo suggerisce che $\mathrm{V_{ref {\_} ext}}$ sarà sempre leggermente maggiore di $\mathrm{V_{out}}$, in quanto esterno allo shunt.
Il $\mathrm{V_{out}}$ realmente prodotto dallo ShuntLDO sarà:
\begin{equation}
  \mathrm{V_{out} = 2 \cdot V_{ref} = 2 \cdot ( V_{ref {\_} ext} - I_{shunt} \cdot R_{gnd} )}
\end{equation}

Questo effetto è visibile in figura \ref{SLDO2Astatic}, dove all'aumentare della corrente $\mathrm{V_{out}}$ si ha una lieve flessione della tensione in uscita. Facendo un fit lineare dei punti si ottiene una pendenza di - 0.015 $\Omega$. 
La pendenza ottenuta dal fit non è unicamente data dai \textit{wire bond}, ma ha un contributo aggiuntivo dovuto alla resistenza delle piste e dei connettori.
Una misura più precisa può essere eseguita sfruttando i pin $\mathrm{V_{out{\_}Sense}}$ e $\mathrm{I_{out {\_} Sense}}$, indicati sulla PCB con P5 e P6, rispettivamente. 
Questi due pin di monitoraggio sono collegati al $\mathrm{V_{out}}$ del regolatore e al suo GND.
Essendo piste in cui non scorre corrente, l'effetto resistivo di \textit{wire bond} è eliminato ed è possibile misurare il valore di tensione del GND locale.

\begin{figure}
\centering
\includegraphics[scale=.4]{Immagini/Viout}
\caption{Andamento della tensione del GND dello shunt rispetto al GND della PCB al variare della corrente di alimentazione del circuito.}
\label{VioutSense}
\end{figure}

Si è proceduto a misurare il valore di tensione sul pin $\mathrm{I_{out{\_}Sense}}$ al variare della corrente di alimentazione $\mathrm{I_{in}}$. 
La misura è stata eseguita per due diversi valori di carico statico: $4 \Omega$ e in assenza di carico per cui il circuito fra $\mathrm{V_{out}}$ e GND è aperto, presentandosi come una resistenza infinita ($\infty$).
%$4 \Omega$ e $\infty$, con il valore infinito si intende la configurazione in cui il carico è assente e dunque il circuito tra $\mathrm{V_{out}}$ e GND è aperto, presentandosi di fatto come una resistenza infinita. 
In figura \ref{VioutSense} sono riportati gli andamenti di $\mathrm{V_{out{\_}Sense}}$ in funzione di $\mathrm{I_{in}}$ per le due configurazioni.
La pendenza delle due curve riportate è la stessa, pari a 4 m$\Omega$, e ci fornisce un'indicazione del valore resistivo, R, dei \textit{wire bond}, mentre l'offset cambia in accordo con quanto ci si aspetterebbe da uno spostamento di tensione dato dalla corrente che scorre verso il GND del chip: maggiore è la corrente richiesta dal carico, minore è l'offset. 
La differenza tra i due offset è di 1 mV che per l'appunto corrisponde alla caduta di tensione causata dalla diversa corrente che scorre nello shunt nei due casi:
\begin{equation}
\mathrm{\Delta V = \Delta I_{shunt} \cdot R_{wire \_ bond} = 0.250 A \cdot 4 m\Omega = 1 mV}
\end{equation}
%In figura \ref{VioutSense} è possibile vedere come le due diverse configurazioni di carico influenzino la misura con un offset. 
%Infatti nei due casi la pendenza è la stessa e dà un'indicazione del valore resistivo R dei wire bond, l'offset invece rispecchia il fatto che lo spostamento di tensione è dato dalla corrente che scorre verso il GND del chip, che nel caso in cui il carico richieda più corrente diminuisce.
Ad esempio ad 1 A con il carico resistivo da 4 $\Omega$, la corrente che effettivamente scorre verso GND nello shunt è 0.75 A (questo nel caso $\mathrm{V_{out} = 1 V}$).

%\par \begin{center} !!! CHE ROBA E'?! CAPTION O ALMENO DESCRIZIONE NEL TESTO !!! \end{center} \par
%
%\begin{center}
%\begin{tabular}{cccc}
%\hline
%$\mathrm{R_{rossa}}$  & $\mathrm{R_{blu}}$ & $\mathrm{Offset_{rosso}}$ & $\mathrm{Offset_{blu}}$\\
%\hline
%4 m$\Omega$ & 4 m$\Omega$ & -1 mV & -2 mV\\
%\hline
%\end{tabular}
%\end{center}

Il valore della resistenza equivalente è molto piccolo e, come detto, dipende in prima approssimazione dal numero di \textit{wire bond}.
Fintantoché lo ShuntLDO è utilizzato come circuito a se stante su una PCB di test, dato che non ci sono problemi di spazio, è possibile utilizzare un gran numero di connessioni per ridurre al minimo differenze tra $\mathrm{GND_{PCB}}$ e $\mathrm{GND_{REG}}$, in modo da rendere trascurabile il fenomeno.

\subsection{Offset}

Torniamo adesso a considerare la possibilità, nella versione da 2 A, di regolare esternamente il $\mathrm{V_{offset}}$. 
Una tensione di offset alta consente di raggiungere il punto di lavoro
\footnote{
  Il regolatore per funzionare al meglio deve trovarsi ad una differenza di tensione di almeno 1.4 V.
}
prima, cioè con un minor consumo di corrente, un valore di $\mathrm{V_{offset}}$ basso ha l'effetto contrario.
Si può verificare questo fenomeno misurando la tensione di uscita del regolatore $\mathrm{V_{out}}$, per diversi valori di $\mathrm{V_{offset}}$, al variare di $\mathrm{I_{in}}$, come mostrato in figura \ref{VoutVsVoffset}.

\begin{figure}
\centering
\includegraphics[scale=.4]{Immagini/VoutVsVoffset}
\caption{Andamento di $\mathrm{V_{out}}$ al variare della corrente in ingresso per differenti valori di $\mathrm{V_{offset}}$.}
\label{VoutVsVoffset}
\end{figure}

Dalla misura di $\mathrm{V_{in}}$ al variare della corrente in ingresso (figura \ref{VinVsVoffset}) è possibile ricavare, con un fit lineare nella regione di funzionamento del circuito, l'intercetta con l'asse y, corrispondente all'offset dello ShuntLDO. 

\begin{figure}
\centering
\includegraphics[scale=.4]{Immagini/VinVsVoffset}
\caption{Andamento di $\mathrm{V_{in}}$ al variare della corrente in ingresso per differenti valori di $\mathrm{V_{offset}}$.}
\label{VinVsVoffset}
\end{figure}

%Questi valori sono riportati nella tabella seguente:
Nella tabella seguente sono riportati i valori ottenuti dal fit delle misure:

\begin{center}
\begin{tabular}{ccc}
\hline
Offset 0.4 V & ffset 0.6 V & Offset 0.8 V\\
\hline
0.352 V & 0.539 V & 0.756 V\\
\hline
\end{tabular}
\end{center}
%\[
%\begin{array}{ccc}
%
%\toprule
%\mathrm{Offset 0.4 V} & \mathrm{Offset 0.6 V} & \mathrm{Offset 0.8 V} \\
%
%\midrule
%
%0.352 V & 0.539 V & 0.756 V \\
%
%\bottomrule
%\end{array}
%\]

Come si può vedere il valore effettivo è sempre leggermente minore di quello impostato. Questo fenomeno è stato osservato anche nelle misure eseguite sullo ShuntLDO presente all'interno del chip RD53A e che descriveremo più avanti.
\FloatBarrier

\subsection{Comportamento dinamico}

\begin{figure}[!htb]
\centering
\includegraphics[scale=.3]{Immagini/SetupScheme}
\caption{Schema del setup per lo studio del comportamento dinamico dello ShuntLDO.}
\label{Setupscheme}
\end{figure}

Fino ad ora abbiamo visto la caratterizzazione con carico statico dello ShuntLDO a 2 A.
E' fonamentale studiare il comportamento dello ShuntLDO in risposta ad una variazione dinamica del carico, focalizzando l'attenzione sulla velocità dello shunt nel riequilibrare il consumo in corrente. 
%Oltre alla caratterizzazione statica un punto fondamentale è studiare il comportamento dello ShuntLDO in risposta ad una variazione dinamica del carico, andando a focalizzare l'attenzione sulla velocità dello shunt nel riequilibrare il consumo in corrente. 
La risposta dinamica dipende da molti fattori, quali il punto di lavoro a cui si trova lo ShuntLDO, il tempo in cui avviene la variazione di carico e l'entità di tale variazione. 
Visto che il circuito è alimentato tramite un generatore di corrente a 1.5 A e $\mathrm{Vref=0.5 A}$, ci aspettiamo una tensione all'uscita di 1 V.
Al fine di introdurre un carico dinamico, in parallelo al carico statico, è stato messo un mosfet in serie ad una resistenza, R5, di $0.1 \Omega$.
La corrente assorbita dal mosfet, indicata con $\mathrm{I_{mosfet}}$, sarà ricavata misurando la caduta di tensione su questa resistenza.
Il mosfet è pilotato tramite un generatore di impulsi.
A seconda dell'ampiezza del segnale inviato al gate del mosfet c'è una maggiore o minore corrente che scorre tra drain e source. 
Il setup di queste misure è rappresentato schematicamente in figura \ref{Setupscheme}: sulla sinistra in verde, la PCB; in grigio sono riportati gli alimentatori; sulla destra l'oscilloscopio con cui sono misurate la tensione in ingresso $\mathrm{V_{in}}$
\footnote{
L'alimentazione dello ShuntLDO è comunque in corrente.
}, rappresentata in giallo, la tensione in uscita $\mathrm{V_{out}}$, in arancione, e le tensioni agli estremi della resistenza R5 in azzurro; infine, sulla sinistra della PCB, collegato al gate del mosfet, è presente un generatore di impulsi. 
Per quanto riguarda la durata dell'impulso da mandare al gate del mosfet, si è scelto di tenere fronte di salita e discesa ben distanti in modo da osservare separatamente gli effetti dovuti all'uno e all'altro.
In una situazione in cui il carico del regolatore è il chip, infatti, le variazioni sarebbero di minor durata rispetto alla lunghezza dell'impulso utilizzato in queste misure.
Ad ogni modo lo scopo di queste misure è solo quello di vedere gli effetti al passaggio da un certo consumo di corrente ad uno maggiore e viceversa ed un impulso di breve durata sovrapporrebbe questi due effetti, non permettendo di valutarne l'effettiva entità, in quanto i contributi sono opposti e, su tempi brevi, si sovrapporrebbero cancellandosi a vicenda.
In queste misure l'attenzione sarà focalizzata su variazioni di $\mathrm{V_{in}}$ e $\mathrm{V_{out}}$ in ampiezza e sul tempo di recupero al variare di $\mathrm{I_{mosfet}}$ per una data $\mathrm{R_{load}}$.

L'introduzione del carico $\mathrm{R_{load}}$ in parallelo alla resistenza connessa a $\mathrm{V_{out}}$ è possibile posizionando un jumper sul pin P10 della PCB.
Già dalle prime misure con l'oscilloscopio è possibile vedere che che l'utilizzo del mosfet come carico dinamico non è esente da fenomeni di alterazione dei segnali che rendendo difficile una loro corretta interpretazione. 
\begin{figure}
\begin{subfigure}{.5\textwidth}
  \centering
  \includegraphics[width=.96\linewidth]{Immagini/zoomTransientTest1}
  \caption{1a}
  \label{TransientTest:sfig1}
\end{subfigure}%
\begin{subfigure}{.5\textwidth}
  \centering
  \includegraphics[width=.95\linewidth]{Immagini/zoomTransientTest2}
  \caption{1b}
  \label{TransientTest:sfig2}
\end{subfigure}
\begin{subfigure}{.95\textwidth}
  \centering
  \includegraphics[width=\linewidth]{Immagini/zoomTransientTest3}
  \caption{1c}
  \label{TransientTest:sfig3}
\end{subfigure}
\caption{Schermata catturata dall'oscilloscopio: in giallo è rappresentata la tensione in ingresso, in arancione quella in uscita e in verde e blu la tensione sui i terminali di R5, la cui differenza è riportata in alto a destra in azzurro. Nella figura in basso, a lato della schermata dell'oscilloscopio, è riportato lo schematico della parte di circuito con mosfet e resistenza.}
\label{TransientTest}
\end{figure}
%\begin{figure}
%\centering
%\includegraphics[scale=.2]{Immagini/TransientTest}
%\caption{Schermata catturata dall'oscilloscopio: in giallo è rappresentata la tensione in ingresso, in arancione quella in uscita e in verde e blu la tensione sui i terminali di R5, la cui differenza è riportata in alto a destra in azzurro. A lato della schermata dell'oscilloscopio è riportato lo schematico della parte di circuito con mosfet e resistenza.} 
%\label{TransientTest}
%\end{figure} 
Prendendo come riferimento la figura \ref{TransientTest}, si può notare un'asimmetria nelle variazioni di $\mathrm{V_{out}}$, in arancione in basso a sinistra, mentre in $\mathrm{V_{in}}$, in giallo in alto a sinistra, la risposta è simmetrica.
In alto a destra, in azzurro, è riportata la differenza tra le tensioni misurate ai capi di R5, tensioni che sono riportate in basso a destra in blu e in verde e da cui è possibile calcolare la corrente che scorre tra Drain e Source del mosfet. 
Inoltre sono visibili delle oscillazioni in corrispondenza dell'istante in cui il mosfet si spegne smettendo di assorbire corrente.
Questo comportamento si riflette su $\mathrm{V_{out}}$ ed è quindi all'origine dell'asimmetria.
Prima di procedere alle misure della risposta dinamica nelle varie combinazioni $\mathrm{I_{mosfet}}$-$\mathrm{R_{load}}$ questo aspetto è stato approfondito, al fine di capirne l'origine, probabilmente un contributo del mosfet non trascurabile.
L'impulso utilizzato in questa prima fase ha:
\begin{itemize}
  \item frequenza di 50Hz;
  \item durata di 3 $\mu$s;
  \item fronte di salita di 40 ns.
\end{itemize}
%In questa prima fase l'impulso utilizzato ha le seguenti caratteristiche: frequenza 50 Hz, durata 3 $\mu$s, durata del fronte di salita 40 ns.

\subsubsection{Contributo mosfet}
\begin{figure}
\centering
\includegraphics[width=\linewidth]{Immagini/MosfetBehaviour}
\caption{Sulla sinistra è riportato la schermata dell'oscilloscopio, che mostra l'andamento della tensione sui terminali di R in funzione del tempo, sulla destra è invece riportato lo schema della modifica al circuito.}
\label{MosfetBehaviour}
\end{figure}

\begin{figure}
\centering
\includegraphics[width=\linewidth]{Immagini/RiseTime}
\caption{In alto a sinistra è riportata la risposta del $\mathrm{V_{out}}$ per impulsi con tempo di salita 40 ns, il basso a destra invece è la risposta a impulsi con tempi di salita di 200ns. Sia per il $\mathrm{V_{out}}$ che per le tensioni su R5 il comportamento migliora rallentando l'impulso.}
\label{RiseTime}
\end{figure}

Per esaminare il comportamento del mosfet in risposta all'impulso mandato sul gate, si è proceduto ad isolare questa parte del circuito dal resto della PCB, connettendo sul pin P10 (connesso al drain del mosfet) una resistenza in serie ad una batteria stilo.
La batteria ricopre il ruolo di $\mathrm{V_{out}}$, mentre la resistenza è necessaria alla misura delle correnti che scorrono nel mosfet ed ha un valore di 2.7 $\Omega$. 



Come si vede dalla figura \ref{MosfetBehaviour}, le oscillazioni in corrispondenza dello spegnimento del mosfet sono presenti anche una volta che questo è stato isolato.
Questo comportamento è segno del fatto che queste oscillazioni sono generate dal mosfet stesso nel momento in cui il canale, che collega drain e source, si interrompe.
Inoltre il fronte di salita è circa 100 ns e non 40 ns come ci aspetteremmo da un mosfet ideale con risposta istantanea.
Esaminando la documentazione del mosfet presente sulla PCB 
%footnote{ZXMN20B28KTC http://www.mouser.com/ds/2/115/ZXMN20B28K-94822.pdf
si può verificare che effettivamente il tempo di "accensione" è superiore a 40 ns (Turn-on rise time 76,9 ns) e, inoltre, sono presenti capacità in ingresso equivalenti a $\mathrm{358 pF}$.
E', quindi, impossibile vedere la risposta dello ShuntLDO a segnali più veloci della risposta del mosfet, per cui le misure successive sono state prese con un tempo di salita del segnale del generatore di impulsi di 200 ns.
La figura \ref{RiseTime} moatra il miglioramento fra la configurazione con tempo di salita del generatore di 40 ns e quella con 200 ns.
In questo modo l'uscita dello ShuntLDO corrisponde ad una simulazione di variazione di carico più lenta, ma meno affetta dalle caratteristiche del mosfet.
%In figura \ref{RiseTime} è visibile come la situazione precedente, in cui l'impulso ha un tempo di salita di 40 ns, migliora visibilmente passando a 200 ns, in questo modo quello che viene simulato all'uscita dello ShuntLDO è un variazione di carico più lenta ma il cui comportamento è affetto in modo minore dalle caratteristiche del mosfet. 

\subsubsection{Misure}

Riportiamo ora le misure di caratterizzazione della tensione di ingresso, $\mathrm{V_{in}}$, e di uscita, $\mathrm{V_{out}}$, per tre differenti valori di $\mathrm{R_{load}}$ e al variare di $\mathrm{I_{mosfet}}$.
I valori di $\mathrm{R_{load}}$ scelti sono stati $1 \Omega$, $2.1 \Omega$ e $4 \Omega$, e dato che $\mathrm{V_{out}=1V}$, in termini di correnti  $\mathrm{I_{load}}$, questi corrispondono rispettivamente a 1 A, 0.475 A e 0.250 A.
Sono stati misurati sia le variazioni in ampiezza che il tempo di recupero di $\mathrm{V_{in}}$ e $\mathrm{V_{out}}$.

La prima differenza notata è il tempo di recupero di $\mathrm{V_{in}}$ e $\mathrm{V_{out}}$: il primo è molto più lungo, dell'ordine dei $\mu$s, e dipendente dall'entità di $\mathrm{I_{mosfet}}$, il secondo ha durata di circa 300 ns indipendentemente dal valore di $\mathrm{I_{mosfet}}$.
Questo comportamento è dovuto al fatto che il riequilibrio della tensione di ingresso dipende anche dal generatore esterno, le cui variazioni sono più lente.
Va ricordato che lo ShuntLDO è alimentato in corrente con 1.5 A, dunque nel momento in cui $\mathrm{I_{load}+I_{mosfet}}$ raggiungono valori vicini o addirittura superiori  a $\mathrm{I_{in}}$, si ha un crollo della tensione in ingresso e del $\mathrm{V_{out}}$, poiché si sta chiedendo allo ShuntLDO di fornire una corrente superiore a quella a sua disposizione.

Per ciascun valore di $\mathrm{R_{load}}$, dunque, è stata fatta variare la corrente assorbita dal mosfet $\mathrm{I_{mosfet}}$ e misurato l'effetto di undershoot e overshoot sulle tensioni di $\mathrm{V_{out}}$ e $\mathrm{V_{in}}$. 
% Con corrente totale si intende la somma di quella assorbita dal mosfet e dalla resistenza di carico.
Le misure eseguite prendono in considerazione anche situazioni in cui la variazione del consumo in corrente eccede l'intervallo fisico di operatività del chip: misure in cui la variazione del carico è il doppio del valore statico hanno interesse nell'ottica di quello che può succedere al momento dell'accensione del chip, le cui variazioni di consumo in regime di lavoro, di norma, non superano i 500 mA.(controllare) 
Come detto in precedenza, gli impulsi utilizzati presentano una durata che consente di differenziare tra gli effetti dovuti al fronte di salita e quelli prodotti dal fronte di discesa. 
Facendo riferimento alla figura \ref{VoutUnd}, si possono vedere, in valore assoluto,gli undershoot della tensione di uscita a cui è applicato il carico in funzione della corrente che scorre nel mosfet (sinistra) e della corrente totale (destra), dove questa è la somma di quella assorbita dal carico e quella del mosfet. 
%I primi risultati riportano gli undershoot della tensione di uscita a cui è applicato il carico, riferendosi ai grafici in figura \ref{VoutUnd} sono riportati i valori assoluti di tali variazioni in funzione della corrente che scorre nel mosfet (sinistra) e della corrente totale (destra), la corrente totale è somma di quella assorbita dal carico e dal mosfet. 
\begin{figure}
\centering
\includegraphics[width=0.9\linewidth]{Immagini/VoutUnd}
\caption{Grafici che riportano l'entità dell'undershoot del $\mathrm{V_{out}}$ in funzione della corrente totale, grafico di sinistra, e della corrente del mosfet, grafico di destra.}
\label{VoutUnd}
\end{figure}
\begin{figure}
\centering
\includegraphics[width=0.9\linewidth]{Immagini/VoutOver}
\caption{Grafici che riportano l'entità dell'overshoot del $\mathrm{V_{out}}$ in funzione della corrente totale, grafico di sinistra, e della corrente del mosfet, grafico di destra.}
\label{VoutOver}
\end{figure}
In blu sono riportate le misure ottenute con un carico resistivo di 1 $\Omega$, in rosso 2.1 $\Omega$ e in verde 4 $\Omega$.
Come si può vedere dal grafico di destra, le tre curve seguono lo stesso andamento, dato che vi è una relazione fra la variazione di $\mathrm{V_{out}}$ e $\mathrm{I_{mosfet}}$, indipendente dal valore della resistenza.
Inoltre, limitandosi ad un intervallo di variazioni di corrente verosimili per il chip, si osservano variazioni relativamente piccole di $\mathrm{V_{out}}$.
Ad esempio, con $\mathrm{I_{mosfet}= 0.4 }$ A, $\mathrm{\Delta V_{out} \simeq 20mV}$.
Lo stesso comportamento è visibile nei grafici di figura \ref{VoutOver}, dove è riportata l'entità delle variazioni di $\mathrm{V_{out}}$ a seguito dello spegnimento del mosfet, cioè l'effetto che si ha sul fronte di discesa dell'impulso. 
%Nell'esaminare questi andamenti va ricordato che la corrente in ingresso al circuito è 1.5 A, quindi punti per i quali si ha una $\mathrm{I_{tot}}$ vicina o superiore a questo valore sono ottenuti in una situazione in cui lo ShuntLDO è impossibilitato a compiere il suo lavoro. 
Ricordiamo che la corrente che passa in R3 è un millesimo di quella che scorre nel ramo in cui si hanno carico e shunt.

Come per il $\mathrm{V_{out}}$ è stato misurato l'undershoot e l'overshoot della tensione in ingresso $\mathrm{V_{in}}$. I grafici degli undershoot sono riportati in figura \ref{VinUnd}, quelli riguardanti gli overshoot in figura \ref{VinOver}. 

\begin{figure}
\centering
\includegraphics[width=0.9\linewidth]{Immagini/VinUnd}
\caption{Grafici che riportano l'entità dell'undershoot del $\mathrm{V_{in}}$ in funzione della corrente totale, grafico di sinistra, e della corrente del mosfet, grafico di destra.}
\label{VinUnd}
\end{figure}

\begin{figure}
\centering
\includegraphics[width=0.9\linewidth]{Immagini/VinOver}
\caption{Grafici che riportano l'entità dell'undershoot del $\mathrm{V_{in}}$ in funzione della corrente totale, grafico di sinistra, e della corrente del mosfet, grafico di destra.}
\label{VinOver}
\end{figure}

In entrambi i casi la variazione della tensione in ingresso dipende sia dalla variazione di corrente $\mathrm{I_{mosfet}}$ che dalla corrente fissa $\mathrm{I_{load}}$. 
Per valori elevati di $\mathrm{I_{mosfet}}$ la tensione in ingresso inizia ad oscillare, con periodi di qualche $\mu$s.
Questo comportamento è dovuto al generatore utilizzato, in particolare l'alimentazione in corrente è stata ottenuta utilizzando un generatore di tensione, ma limitando la corrente in uscita. 
Nel momento in cui si ha una variazione di carico molto veloce che provoca un abbassamento di $\mathrm{V_{out}}$, si ha una piccola ripercussione sulla tensione di ingresso: dato che il generatore è di tensione, limitato in corrente, per tenere costante $\mathrm{I_{in}}$ avrà un abbassamento di tensione, ma con tempi più lunghi rispetto a quelli con cui lo ShuntLDO riesce a riequilibrare $\mathrm{V_{out}}$.
Il comportamento oscillatorio di $\mathrm{V_{in}}$, che compare quando $\mathrm{I_{tot}}$ è intorno al valore massimo, $\mathrm{I_{in}}$, ha permesso di constatare come fluttuazioni della tensione in ingresso non influiscano sulla tensione generata dal regolatore.
Questo può essere visto bene utilizzando l'oscilloscopio: in figura \ref{DipVoutVin} è mostrato uno screenshoot in cui è riportato, in giallo, l'andamento della tensione in ingresso in funzione del tempo, e, in arancione, la tensione di $\mathrm{V_{out}}$.
Si nota che le scale di tempo di recupero sono differenti, i.e. alcuni $\mu$s per $\mathrm{V_{in}}$ e circa 300 ns per $\mathrm{V_{out}}$.

\begin{figure}
\begin{subfigure}{.5\textwidth}
  \centering
  \includegraphics[width=.95\linewidth]{Immagini/zoomDipendenzaVoutdaVin1}
  \caption{1a}
  \label{DipVoutVin:sfig1}
\end{subfigure}%
\begin{subfigure}{.5\textwidth}
  \centering
  \includegraphics[width=.95\linewidth]{Immagini/zoomDipendenzaVoutdaVin2}
  \caption{1b}
  \label{DipVoutVin:sfig2}
\end{subfigure}
\caption{Differenze nei tempi di recupero tra $\mathrm{V_{out}}$, in giallo, e $\mathrm{V_{out}}$, in arancione.}
\label{DipVoutVin}
\end{figure}
%\begin{figure}
%\centering
%\includegraphics[scale=.35]{Immagini/DipendenzaVoutdaVin}
%\caption{Differenze nei tempi di recupero tra $\mathrm{V_{out}}$, in giallo, e $\mathrm{V_{out}}$, in arancione.}
%\label{DipVoutVin}
%\end{figure}

\subsubsection{Serie di due ShuntLDO}

Dato che eventuali oscillazioni della tensione in ingresso causerebbero oscillazioni di tensione in tutta la catena di moduli, è importante che queste non si ripercuotano sul $\mathrm{V_{out}}$. 
Per verificare questo aspetto si è monitorata la tensione di uscita di uno ShuntLDO messo in serie con un secondo a cui è stato applicato un carico variabile, tramite l'utilizzo del mosfet, come già visto nelle misure precedenti.
In figura \ref{SLDOserie} sono affiancati uno schema del setup (sinistra) e la foto dei due ShuntLDO in serie (destra). 
Il serie di due ShuntLDO, entrambi con un carico statico di 4 $\Omega$, è stato alimentato con una corrente in ingresso di 1.5 A.
Sul secondo ShuntLDO è stato collegato l'impulsatore che regola l'assorbimento di corrente  da parte del mosfet. 
Monitorando con l'oscilloscopio la tensione di $\mathrm{V_{out}}$ di entrambi e il $\mathrm{V_{in}}$ del primo shunt della catena (quello con solo carico statico) è stato possibile verificare come le fluttuazioni di tensione non influenzino la generazione della tensione di $\mathrm{V_{out}}$.
\begin{figure}[h!]
\centering
\includegraphics[scale=.30]{Immagini/SLDOserie}
\caption{Sulla destra foto dei due ShuntLDO in serie di cui a sinistra è riportato uno schema delle connessioni con generatore e impulsatore.}
\label{SLDOserie}
\end{figure}
In particolare in figura \ref{ScreenSerie} si vede che, nonostante il secondo ShuntLDO sia al limite, la generazione di $\mathrm{V_{out}}$ da parte del secondo non ha ripercussioni. 
Il campionamento dei segnali mostrati, acquisiti con l'oscilloscopio, è stato ottenuto con una $\mathrm{I_{mosfet}}$ di 1.2 A, corrispondenti ad una $\mathrm{I_{tot}}$ di circa 1.45 A, quindi molto vicino al limite di 1.5 A. 
\begin{figure}[h!]
\centering
\includegraphics[scale=.32]{Immagini/ScreenSerie}
\caption{Schermata dell'oscilloscopio in cui è riportata in giallo la tensione in ingresso al primo ShuntLDO della catena,  in celeste la tensione di $\mathrm{V_{out}}$ sempre dello ShuntLDO1 e in verde la tensione di $\mathrm{V_{out}}$ dello ShuntLDO2 su cui è applicato il carico dinamico. Le fluttuazioni in tensione originate dalla variazione di carico sullo ShuntLDO 2 si ripercuotono sul $\mathrm{V_{in}}$ dello ShuntLDO1 (giallo) ma non sulla tensione da esso generata (celeste).}
\label{ScreenSerie}
\end{figure}
In verde è riportato l'andamento di $\mathrm{V_{out}}$ del secondo ShuntLDO, che mostra importanti undershoot e overshoot, i quali, a loro volta e come visto in precedenza, causano come visto in precedenza, fluttuazioni della tensione in ingresso. 
La tensione di ingresso dello ShuntLDO2 corrisponde alla terra della PCB su cui si trova lo ShuntLDO1.
Le visibili fluttuazioni di questa, quindi, si ripercuotono su ShuntLDO2.
Nonostante ciò, come risultato importante, si può notare che $\mathrm{V_{out}}$ del primo ShuntLDO sia indipendente da queste.
