%\frontmatter
\chapter{Introduzione}
%con lo scopo di indagare nuova fisica progetto HL-LHC
In vista della fase-2 di LHC (\textit{Large Hadron Collider}) il tracciatore di CMS dovrà essere completamente sostituito causa danneggiamento da radiazione e peggioramento delle prestazioni dovuto all'alto Pile Up che si avrà nella fase ad alta luminosità di LHC.
La presenza di materiale inerte riduce le prestazioni del rivelatore, per questo motivo già tra fase-0 e fase-1 si è passati ad un sistema di alimentazione \textit{Point of Load}, con l'utilizzo di convertitori DC-DC. Questi convertitori non sono però idonei alla fase di HL-LHC, sia perché  a causa dell'alta radiazione dovrebbero essere posti almeno a distanze di 20--25 cm dall'asse dei fasci, sia per le dimensioni, i convertitori DC-DC richiedono induttanze avvolte su nuclei ferromagnetici, scarsamente miniaturizzabili. 
L'unico approccio che è stato valutato compatibile con il nuovo tracciatore interno di fase-2 è uno schema di alimentazione seriale. 
Questa rappresenta una novità ed una sfida senza precedenti, in quanto è la prima volta che verrebbe utilizzato su larga scala in un rivelatore per la fisica di alte energie, che opera in condizioni limite. 
Per gestire l'alimentazione seriale localmente sui singoli ROC (\textit{Read Out Chip}) è stato sviluppato un circuito denominato ShuntLDO, che consente una regolazione fine delle tensioni e assicura un consumo di corrente costante, indipendente dal consumo del ROC.

Nel presente lavoro di tesi ho studiato il circuito di ShuntLDO seguendone gli sviluppi, partendo da un primo prototipo in tecnologia CMOS a $65 \nm$ capace di gestire correnti da $0.5\A$, fino ad arrivare allo studio dello ShuntLDO da $2\A$  posto all'interno del prototipo del nuovo ROC, RD53A. 
I punti di interesse sono la verifica di affidabilità del principio di funzionamento con cui è progettato lo ShuntLDO, descritto nel capitolo~\ref{AlimentazioneSeriale}, e la messa in luce delle criticità che si possono avere, specialmente in fasi delicate quali ad esempio l'accensione.

