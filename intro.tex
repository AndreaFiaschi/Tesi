%\frontmatter
\chapter{Introduzione}

Espongo in questa tesi una prima campagna di validazione dell'alimentazione seriale basata su ShuntLDO. Questa \`e una tecnologia innovativa e di frontiera, mai utilizzata prima d'ora in un esperimento di fisica delle alte energie, in fase di sviluppo per l'Inner Tracker, il rivelatore interno a pixel del nuovo Tracciatore di CMS per HL-LHC. 

Nonostante l'impiego di tecnologia CMOS a $65\nm$ con alimentazione ${\cal O}(1\V)$, il consumo di potenza dell'elettronica di prossimit\`a \`e enorme a causa dell'elevato rate ($3\GHz/\mathrm{cm^2}$), dovuto alla luminosit\`a istantanea ($\mathrm{5-7.5\cdot 10^{34}cm^{-2}s^{-1}}$), e il basso rumore, indispensabile per le basse soglie necessarie per mitigare gli effetti del danneggiamento da radiazione (fino a $\mathrm{2.3\cdot10^{16} n_{1\MeV\, eq}/cm^2}$, $1.2\Grad$). L'alimentazione seriale \`e l'unico schema possibile per il funzionamento dei due miliardi di canali dell'Inner Tracker perché consente di trasportare i $50-60\kW$ di potenza, necessari al funzionamento, ad una tensione pi\`u elevata, ${\cal O}(10\V)$, concatenando una decina di moduli a pixel.

% che consenta per. Questa enorme potenza deve essere quindi portata al rivelatore ad, prevedendo una qualche forma di conversione locale. 
%senza che le prestazioni di questo apparato ne escano irrimediabilmente ridimensionate.

Altri dispositivi di conversione locale, come ad esempio i convertitori DC-DC,
%scelti per le regioni pi\`u esterne,
%avrebbero un impatto non marginale sul materiale passivo,
non possono essere utilizzati a causa della elevatissima radiazione e del ridotto spazio disponibile nel rivelatore, un cilindro di $\sim5\m$ di lunghezza intorno alla \textit{beam pipe} con un raggio compreso tra $\sim20$ e $\sim 30\cm$. L'alimentazione diretta richiederebbe, invece, un materiale passivo, in cavi, circa dieci volte maggiore. 
Ciò vanificherebbe i miglioramenti di prestazioni che questo nuovo rivelatore \`e chiamato ad avere per l'ambizioso programma di HL-LHC, focalizzato sulle misure di precisione nel settore di Higgs e sulle ricerche di nuova fisica oltre il Modello Standard. 

%Una delle maggiori limitazioni del tracciatore attualmente istallato in CMS \`e, infatti, il materiale passivo e tra i requisiti del Tracciatore per HL-LHC c'\`e la sua minimizzazione per non vanificare i miglioramenti di prestazioni che questo nuovo rivelatore \`e chiamato ad avere per l'ambizioso programma di fisica di HL-LHC, focalizzato sulla fisica di precisione del bosone di Higgs e sulle ricerche di nuova fisica. 

%-----

Il cuore dello schema di alimentazione seriale \`e lo ShuntLDO, uno speciale circuito con un regolatore \textit{Low Drop Out} accoppiato ad uno shunt. Lo ShuntLDO garantisce che l'elemento nella catena seriale presenti al sistema di alimentazione remoto un carico fisso, indipendente dal consumo istantaneo del ROC (\textit{Readout Chip}).
% e schematizzabile come una resistenza in serie ad una tensione di offset. Questo comportamento rendo lo ShuntLDO molto versatile e utilizzabile in parallelo (come necessario in un ROC che richiede una alimentazione analogica e una digitale).
%Questo carico varia a seconda della corrente che scorre nella catena seriale ed è caratterizzato come un offset di tensione $\mathrm{V_{offset}}$ più una resistenza caratteristica R.


%con lo scopo di indagare nuova fisica progetto HL-LHC
% In vista della fase-2 di LHC (\textit{Large Hadron Collider}) il tracciatore di CMS dovrà essere completamente sostituito causa danneggiamento da radiazione e peggioramento delle prestazioni dovuto all'alto Pile Up che si avrà nella fase ad alta luminosità di LHC.
% La presenza di materiale inerte riduce le prestazioni del rivelatore, per questo motivo già tra fase-0 e fase-1 si è passati ad un sistema di alimentazione \textit{Point of Load}, con l'utilizzo di convertitori DC-DC. Questi convertitori non sono però idonei alla fase di HL-LHC, sia perché, a causa dell'alta radiazione, dovrebbero essere posti almeno a distanze di 20--25 cm dall'asse dei fasci, sia per le dimensioni, i convertitori DC-DC sono scarsamente miniaturizzabili. 
% L'unico approccio che è stato valutato compatibile con il nuovo tracciatore interno di fase-2 è uno schema di alimentazione seriale. 
% Questa rappresenta una novità ed una sfida senza precedenti, in quanto è la prima volta che verrebbe utilizzato su larga scala in un rivelatore per la fisica di alte energie, che opera in condizioni limite. 
% Per gestire l'alimentazione seriale localmente sui singoli ROC (\textit{Readout Chip}) è stato sviluppato un circuito denominato ShuntLDO, che consente una regolazione fine delle tensioni e assicura un consumo di corrente costante, indipendente dal consumo del ROC.

% Nel presente lavoro di tesi ho studiato il circuito di ShuntLDO seguendone gli sviluppi, partendo da un primo prototipo in tecnologia CMOS a $65 \nm$ capace di gestire correnti massime di $0.5\A$, fino ad arrivare allo studio dello ShuntLDO a $2\A$  posto all'interno del prototipo del nuovo ROC, RD53A. 
% I punti di interesse sono la verifica di affidabilità del principio di funzionamento con cui è progettato lo ShuntLDO, descritto nel capitolo~\ref{AlimentazioneSeriale}, e la messa in luce delle criticità che si possono avere, specialmente in fasi delicate quali ad esempio l'accensione.

Dopo una breve introduzione su LHC e l'esperimento CMS, rispettivamente nei Capitoli~\ref{ch:LHC} e~\ref{ch:CMS}, nel Capitolo~\ref{AlimentazioneSeriale} sono illustrati il concetto di alimentazione seriale e il principio di funzionamento dello ShuntLDO. L'attivit\`a portata avanti nell'ambito di questo lavoro di tesi, effettuata presso l'INFN di Firenze e il CERN di Ginevra, \`e descritta e discussa nel  Capitolo~\ref{ch:caratShuntLDO}, dedicato alla caratterizzazione dei prototipi di ShuntLDO realizzati in tecnologia CMOS a $65\nm$, e nel Capitolo~\ref{cap:RD53A}, riservato allo studio dello ShuntLDO presente sul primo ROC prototipo, RD53A, sviluppato per HL-LHC dalla collaborazione RD53.

I risultati che ho ottenuto non evidenziano criticit\`a della tecnologia e permettono di proseguire lo sviluppo per giungere al progetto finale.

%I Capitoli~\ref{ch:LHC} e~\ref{ch:CMS} sono dedicati a una breve introduzione su LHC e HL-LHC e dell'esperimento CMS, rispettivamente. Ho quindi descritto il concetto di alimentazione seriale e il principio di funzionamento dello ShuntLDO nel Capitolo~\ref{AlimentazioneSeriale}. 


%I primi risultati non evidenziano criticit\`a della tecnologia, che \`e una scelta praticamente obbligata, e permettono di portarne avanti lo sviluppo e la verifica pi\`u approfondita.

