\chapter{Conclusioni}


Lo sviluppo del regolatore ShuntLDO è motivato dalla necessità di rendere più efficiente la distribuzione di potenza nel rivelatore a pixel degli esperimenti CMS e ATLAS, gli attuali tracciatori sono composti da milioni di canali, raggruppati in moduli. 
A causa dele dimensioni del rivelatore, i moduli devono essere alimentati da cavi con una sezione limitata e una lunghezza tra gli 80 e i 100 metri, questo insieme al consumo di corrente tra 1 e 4 A per modulo crea delle importanti perdite di potenza nei cavi, che superano il consumo del rivelatore stesso. 
L'impiego di cavi più spessi non è possibile, in quanto all'interno del rivelatore si deve evitare interazioni delle particelle con materiale inattivo, che causa un peggioramento delle prestazioni del rivelatore. 
L'utilizzo del regolatore con SLDO permette l'utilizzo di più elementi in parallelo all'interno del modulo incrementando  l'affidabilità del sistema di alimentazione seriale(più elementi in parallelo meno problemi in caso di failure?). 
L'utilizzo di più elementi in parallelo all'interno della linea seriale conferisce  flessibilità al sistema , poiché nonostante siano in parallelo possono generare differenti tensioni in uscita. 
Lo studio di questo prototipo ha confermato l'affidabilità del principio di funzionamento con cui è stato progettato lo SLDO, ha evidenziato gli aspetti cruciali per avere un bilanciamento delle correnti durante l'accensione e ha evidenziato la capacità di isolare il carico dai disturbi esterni presenti sulla linea di alimentazione.
 





