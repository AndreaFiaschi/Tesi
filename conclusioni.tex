\chapter{Conclusioni}


Lo sviluppo del regolatore ShuntLDO è motivato dalla necessità di rendere più efficiente la distribuzione di potenza nel rivelatore a pixel degli esperimenti CMS e ATLAS.
A causa del poco spazio disponibile all'interno del tracciatore, i moduli devono essere alimentati da cavi con una sezione limitata. Inoltre, dato che gli alimentatori di back end sono posti ad una certa distanza dall'esperimento, i cavi hanno una lunghezza tra gli 40 e i 50 metri, questo insieme al consumo di corrente, tra 1 e 4 A per modulo, crea delle importanti perdite di potenza nei cavi, che superano il consumo del rivelatore stesso. 
L'impiego di cavi con una sezione maggiore è problematico, in quanto all'interno del rivelatore si deve evitare il più possibile che vi sia materiale inattivo, causa di peggioramento nelle prestazioni del rivelatore. 
Con il passaggio ad una alimentazione seriale è possibile ridurre le sezioni dei cavi di alimentazione, ma allo stesso tempo vengono introdotte nuove criticità, che è possibile affrontare con l'utilizzo del regolatore ShuntLDO per l'alimentazione dei singoli ROC all'interno dei moduli.
Lo ShuntLDO permette di installare più elementi in parallelo all'interno del modulo incrementando  l'affidabilità del sistema di alimentazione seriale %(più elementi in parallelo meno problemi in caso di guasti)
e conferendo  flessibilità al sistema. 
Alla base del principio di ShuntLDO vi è la necessità di presentare al sistema di alimentazione un carico fisso, indipendentemente dai consumi del ROC. 
Questo carico varia a seconda della corrente che scorre nella catena seriale ed è caratterizzato come un offset di tensione $\mathrm{V_{offset}}$ più una resistenza caratteristica R.
Nel presente lavoro di tesi mi sono concentrato sullo studio del circuito di ShuntLDO sia come elemento a sé stante sia all'interno del ROC RD53A, prototipo del nuovo chip di lettura per il tracciatore di fase-2 di CMS e di ATLAS. 

Nel capitolo~\ref{AlimentazioneSeriale}, dopo una prima introduzione del concetto di ShuntLDO, ho presentato i risultati e le osservazioni ottenuti dallo studio fatto sulle schede di test con i prototipi a 0.5 e 2 A. 
In particolare nelle sottosezioni \ref{EffettiSpuri} e \ref{Voffset}, riguardanti lo ShuntLDO da 0.5 A, ho messo in luce l'importanza di avere delle tensioni di riferimento stabili per un corretto funzionamento del circuito e la capacità dello ShuntLDO di tenere le fluttuazioni del $\mathrm{V_{out}<5 \mV}$ fin tanto che la somma della corrente assorbita dal carico statico e dal carico variabile si mantiene inferiore alla corrente con cui il circuito è alimentato. 
Allo stesso modo per lo ShuntLDO da 2 A, dopo una prima discussione su vari aspetti, quali differenze di GND tra ShuntLDO e scheda di test; regolazione dell'offset e contributo del mosfet alle misure con carico dinamico, ho riportato i risultati dei test, che evidenziano come nell'intervallo tipico di lavoro lo ShuntLDO da 2 A riesca a mantenere le fluttuazioni sul $\mathrm{V_{out}}$ relativamente piccole. 
Le fluttuazioni maggiori si hanno quando le variazioni di corrente assorbita dal carico sono 2--3 volte la corrente assorbita dal carico statico, questo caso fisicamente può corrispondere all'accensione del ROC e va tenuto di conto (le fluttuazioni misurate sono di $\sim 100 \mV$). 
Infine ho messo in evidenza come in un serie di due ShuntLDO le fluttuazioni che si hanno sulla catena non sono viste dal carico applicato allo stesso.

Nel capitolo~\ref{cap:RD53A} ho presentato i risultati dello studio dello ShuntLDO presente nel ROC RD53A. In questo caso va precisato che ci sono due circuiti di alimentazione, uno per la regione analogica del ROC e uno per quella digitale, vi sono dunque due ShuntLDO in parallelo. 
%Lo studio dei due ShuntLDO all'internio di RD53A ha presentato una complicazione?!???
Dal confronto degli andamenti, sezione~\ref{MisureStaticheRD53A} delle tensioni in ingresso, uscita e di riferimento per varie configurazioni di alimentazione (indipendente, parallelo, tensioni di riferimento interne/esterne...) al variare della corrente in ingresso, ho potuto mettere in evidenza vari aspetti. 
La configurazione con cui si alimenta il parallelo dei due ShuntLDO influenza i risultati, in parte a causa di un non perfetto funzionamento dei circuiti responsabili della generazione delle tensioni di riferimento (bandgap), che causa di differenze nell'accensione dei due ShuntLDO e una differente distribuzione delle correnti tra parte analogia e digitale. 
Queste stesse misure hanno evidenziato un sistematico eccesso nel valore della resistenza caratteristica ed un offset inferiore  rispetto ai valori attesi, ciò è indice di resistenze spurie all'interno del circuito per quanto riguarda la resistenza, mentre per l'offset ancora non si è trovato la causa precisa, questo aspetto è tutt'ora indagato. 
Nella sottosezione~\ref{VariazioniCarico} ho verificato la risposta a variazioni di carico, eseguite in modo statico, che hanno evidenziato una dipendenza del $\mathrm{V_{out}}$ dalla corrente assorbita dal carico. %In seguito ho verificato che in fase di accensione non vi fossero oscillazioni

Infine, con un confronto tra alimentazione con ShuntLDO e con il solo regolatore LDO, ho voluto mettere in evidenza l'importanza dello Shunt nella gestione delle fluttuazioni del carico. 
Come ultimo test ho utilizzato il sistema di acquisizione di Bonn per ricavare le distribuzioni di rumore del front end lineare e poter verificare che non vengono influenzate dalla presenza di altri elementi in parallelo o in serie nella catena di alimentazione. 

Lo studio di questo prototipo ha confermato l'affidabilità del principio di funzionamento con cui è stato progettato lo ShuntLDO, ha evidenziato gli aspetti cruciali per avere un bilanciamento delle correnti durante l'accensione e ha dimostrato la capacità di isolare il carico dai disturbi esterni presenti sulla linea di alimentazione.
Allo stesso modo sono state messe in luce criticità e problemi dell'implementazione dello ShuntLDO all'interno del ROC RD53A, tutt'ora sotto studio. Tra queste spiccano l'importanza di un circuito che generi le tensioni di riferimento in modo affidabile, la presenza di resistenze spurie e una dipendenza del valore dell'offset dal valore della resistenza caratteristica. 





