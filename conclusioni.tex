\chapter{Conclusioni}

L'alimentazione seriale basata su ShuntLDO, il nuovo schema di alimentazione scelto per l'Inner Tracker di CMS per HL-LHC, \`e sottoposta ad una estesa verifica dato che non \`e mai stata utilizzata in un esperimento di alte energie. In questo lavoro di tesi, svolto tra l'INFN, Sezione di Firenze, e il CERN, Ginevra, Svizzera, che si inserisce in questa attivit\`a, ho  caratterizzato i prototipi di ShuntLDO realizzati in tecnologia CMOS a $65\nm$, con dimensionamento da $0.5\A$ e $2\A$, e lo ShuntLDO presente sul primo ROC prototipo, RD53A, sviluppato per HL-LHC dalla collaborazione RD53.

%Seguono quindi i Capitoli che descrivono il mio lavoro di tesi effettuato presso l'INFN, Sezione di Firenze, e presso il CERN, Ginevra, Svizzera.

%I Capitoli~\ref{ch:LHC} e~\ref{ch:CMS} sono dedicati a una breve introduzione su LHC e HL-LHC e dell'esperimento CMS, rispettivamente. Ho quindi descritto il concetto di alimentazione seriale e il principio di funzionamento dello ShuntLDO nel Capitolo~\ref{AlimentazioneSeriale}. 


Nel Capitolo~\ref{ch:caratShuntLDO} illustro i risultati e le osservazioni ottenuti dalle prove sulle schede di test che ospitano i prototipi ShuntLDO a $0.5$ e $2\A$. 
In particolare nelle Sezioni~\ref{EffettiSpuri} e~\ref{Voffset}, relative allo ShuntLDO da $0.5\A$, ho messo in luce l'importanza delle tensioni di riferimento per un corretto funzionamento del circuito e la capacità dello ShuntLDO di mantenere stabile entro pochi $\mV$ la tensione regolata $\mathrm{V_{out}}$. 
Questo fin tanto che la corrente assorbita dal carico, anche in condizioni dinamiche, si mantiene inferiore alla corrente che circola nella catena seriale.

Allo stesso modo ho riportato i risultati dei test che evidenziano come nell'intervallo tipico di lavoro lo ShuntLDO da $2\A$ riesca a mantenere le fluttuazioni di $\mathrm{V_{out}}$ relativamente piccole. Queste misure sullo ShuntLDO da $2\A$ hanno richiesto l'approfondimento di alcuni effetti sistematici della scheda di test essenzialmente derivanti dal maggior dimensionamento in corrente di questo prototipo. 
Inoltre, ho effettuato test con due ShuntLDO da $2\A$ posti in serie verificando come lo ShuntLDO sia perfettamente in grado di isolare i carichi a valle della regolazione (i ROC nell'allestimento finale) dalle potenziali fluttuazioni a livello della catena di alimentazione seriale.

Il Capitolo~\ref{cap:RD53A} \`e dedicato alle misure sui due ShuntLDO, necessari ad alimentare parte digitale e analogica, presenti sul prototipo RD53A del ROC per l'IT, che \`e disponibile da poche settimane.
Dal confronto degli andamenti delle tensioni in ingresso, uscita e di riferimento, sezione~\ref{MisureStaticheRD53A}, per varie configurazioni di alimentazione 
al variare della corrente in ingresso, ho potuto mettere in evidenza vari aspetti rilevanti. Tra questi si \`e dimostrato come sia di cruciale importanza la sinergia dello ShuntLDO con la circuiteria ancillare necessaria al suo funzionamento, come per esempio quella dedicata alla generazione delle tensioni di riferimento, specialmente nella fase critica dell'accensione che deve essere esente da situazioni incontrollate. Problemi nell'accensione possono portare il ROC in uno stato che non permette la comunicazione con il sistema di acquisizione dati.

Complessivamente anche il ROC RD53A, per quanto riguarda l'ambito degli ShuntLDO e delle alimentazioni, non ha mostrato particolari criticità ma, al contrario, lo ShuntLDO si \`e dimostrato funzionare entro specifica. Le misure di rumore che ho poi effettuato, confrontando varie tipologie di alimentazione, hanno confermato che l'utilizzo dell'alimentazione seriale non influenza le prestazioni del ROC.

In generale questa prima campagna di misure conferma la validità del concetto di alimentazione seriale e del cuore della sua implementazione, lo ShuntLDO. Nessun problema di base \`e stato riscontrato. Ulteriori verifiche sono necessarie e in corso, non ultime quelle dopo irraggiamento. Particolare cura dovr\`a essere dedicata al disegno di tutta la circuiteria addizionale per evitare problematiche nella fase, critica, di accensione. Questo dovrà poi essere oggetto di estesi test di sistema in condizioni realistiche.


%%%% Archivio pezzi commentati

% Questo lavoro di tesi in questa tesi una prima campagna di validazione dell'alimentazione seriale basata su ShuntLDO. Questa \`e una tecnologia innovativa e di frontiera, mai utilizzata prima d'ora in un esperimento di fisica delle alte energie, in fase di sviluppo per l'Inner Tracker, il rivelatore interno a pixel del nuovo Tracciatore di CMS per HL-LHC. I primi risultati non evidenziano criticit\`a della tecnologia, che \`e una scelta praticamente obbligata, e permettono di portarne avanti lo sviluppo e la verifica pi\`u approfondita.

% L'elevato rate ($3\GHz/\mathrm{cm^2}$), dovuto alla luminosit\`a istantanea (fino a $\mathrm{7.5\cdot 10^{34}cm^{-2}s^{-1}}$), e il basso rumore, dovuto alle basse soglie necessarie per mitigare gli effetti del danneggiamento da radiazione (fino a $\mathrm{2.3 [1\MeV neq]/cm^2}$, $1.2\Grad$), richiedono una elettronica di prossimit\`a con consumi di potenza relativamente elevati nonostante l'impiego di tecnologia CMOS a 65nm che ha tensioni tipiche di alimentazione ${\cal O}(1\V)$. L'alimentazione seriale \`e l'unico schema possibile per il funzionamento dei due miliardi di canali dell'Inner Tracker perch\'e consente di trasportare all'Inner Tracker i $50-60\kW$ di potenza necessari a una tensione pi\`u elevata, ${\cal O}(10\V)$, concatenando una decina di moduli a pixel.

% % che consenta per. Questa enorme potenza deve essere quindi portata al rivelatore ad, prevedendo una qualche forma di conversione locale. 
% %senza che le prestazioni di questo apparato ne escano irrimediabilmente ridimensionate.

% L'alimentazione diretta richiederebbe, invece, un materiale passivo, in cavi, circa dieci volta maggiore. Altri dispositivi di conversione locale, come ad esempio i DC-DC converter scelti per le regioni pi\`u esterne, avrebbero un impatto non marginale sul materiale passivo, ma non possono comunque essere utillizati a causa della elevatissima radiazione e del ridotto spazio disponibile nel limitato volume del rivelatore, un cilindro di $~5m$ di lunghezza intorno alla beam pipe con un raggio compreso tra $\sim20$ e $\sim 30\cm$.

% Una delle maggiori limitazioni del tracciatore attualmente istallato in CMS \`e, infatti, il materiale passivo e tra i requisiti del Tracciatore per HL-LHC c'\`e la sua minimizzazione per non vanificare i miglioramenti di prestazioni che questo nuovo rivelatore \`e chiamato ad avere per l'ambizioso programma di fisica di HL-LHC, focalizzato sulla fisica di precisione del bosone di Higgs e sulle ricerche di nuova fisica. 

% %-----

% Il cuore dello schema di alimentazione seriale \`e lo ShuntLDO, uno speciale circuito con un regolatore Low-Drop-Out e uno shunt. Lo ShuntLDO garantisce che l'elemento nella catena seriale presenti al sistema di alimentazione remoto un carico fisso indipendente dal consumo istantaneo del ROC e schematizzabile come una resistenza in serie ad una tensione di offset. Questo comportamento rendo lo ShuntLDO molto versatile e utilizzabile in parallelo (come necessario in un ROC che richiede una alimentazione analogica e una digitale).
% %Questo carico varia a seconda della corrente che scorre nella catena seriale ed è caratterizzato come un offset di tensione $\mathrm{V_{offset}}$ più una resistenza caratteristica R.

%RD53A sia come elemento circuitale a se stante, Nel presente lavoro di tesi mi sono concentrato sullo studio del circuito di ShuntLDO sia come elemento a sé stante sia all'interno del ROC RD53A, prototipo del nuovo chip di lettura per il tracciatore di fase-2 di CMS e di ATLAS. 

%Lo sviluppo del regolatore ShuntLDO è motivato dalla necessità di rendere più efficiente la distribuzione di potenza nel rivelatore a pixel degli esperimenti CMS e ATLAS.

%A causa del poco spazio disponibile all'interno del tracciatore, i moduli devono essere alimentati da cavi con una sezione limitata. Inoltre, dato che gli alimentatori di back end sono posti ad una certa distanza dall'esperimento, i cavi hanno una lunghezza tra gli 40 e i 50 metri, questo insieme al consumo di corrente, tra 1 e 4 A per modulo, crea delle importanti perdite di potenza nei cavi, che superano il consumo del rivelatore stesso. 
%L'impiego di cavi con una sezione maggiore è problematico, in quanto all'interno del rivelatore si deve evitare il più possibile che vi sia materiale inattivo, causa di peggioramento nelle prestazioni del rivelatore. 
%Con il passaggio ad una alimentazione seriale è possibile ridurre le sezioni dei cavi di alimentazione, ma allo stesso tempo vengono introdotte nuove criticità, che è possibile affrontare con l'utilizzo del regolatore ShuntLDO per l'alimentazione dei singoli ROC all'interno dei moduli.
%Lo ShuntLDO permette di installare più elementi in parallelo all'interno del modulo incrementando  l'affidabilità del sistema di alimentazione seriale %(più elementi in parallelo meno problemi in caso di guasti)
%e conferendo  flessibilità al sistema. 
%Alla base del principio di ShuntLDO vi è la necessità di presentare al sistema di alimentazione un carico fisso, indipendentemente dai consumi del ROC. 
%Questo carico varia a seconda della corrente che scorre nella catena seriale ed è caratterizzato come un offset di tensione $\mathrm{V_{offset}}$ più una resistenza caratteristica R.

%Nel presente lavoro di tesi mi sono concentrato sullo studio del circuito di ShuntLDO sia come elemento a sé stante sia all'interno del ROC RD53A, prototipo del nuovo chip di lettura per il tracciatore di fase-2 di CMS e di ATLAS. 


%La configurazione con cui si alimenta il parallelo dei due ShuntLDO influenza i risultati, in parte a causa di un non perfetto funzionamento dei circuiti responsabili della generazione delle tensioni di riferimento (bandgap), che causa di differenze nell'accensione dei due ShuntLDO e una differente distribuzione delle correnti tra parte analogica e digitale. 
%Queste stesse misure hanno evidenziato un sistematico eccesso nel valore della resistenza caratteristica ed un offset inferiore  rispetto ai valori attesi, ciò è indice di resistenze spurie all'interno del circuito per quanto riguarda la resistenza, mentre per l'offset ancora non si è trovato la causa precisa, questo aspetto è tutt'ora indagato. 
%Nella Sezione~\ref{VariazioniCarico} ho verificato la risposta a variazioni di carico, eseguite in modo statico, che hanno evidenziato una dipendenza del $\mathrm{V_{out}}$ dalla corrente assorbita dal carico. %In seguito ho verificato che in fase di accensione non vi fossero oscillazioni

%Infine, con un confronto tra alimentazione con ShuntLDO e con il solo regolatore LDO, ho voluto mettere in evidenza l'importanza dello Shunt nella gestione delle fluttuazioni del carico. 
%Come ultimo test ho utilizzato un sistema di acquisizione dedicato di Bonn per ricavare le distribuzioni di rumore del front end lineare e poter verificare che non vengono influenzate dalla presenza di altri elementi in parallelo o in serie nella catena di alimentazione. 

%Le prove che ho effettuato con il carico dinamico, effettuate con il 
%hanno richiesto  dopo una prima discussione su vari aspetti, quali differenze di GND tra ShuntLDO e scheda di test; regolazione dell'offset e contributo del mosfet alle misure con carico dinamico, 
%Le fluttuazioni maggiori si hanno quando le variazioni di corrente assorbita dal carico sono 2--3 volte la corrente assorbita dal carico statico, questo caso fisicamente può corrispondere all'accensione del ROC e va tenuto di conto (le fluttuazioni misurate sono di $\sim 100 \mV$). 
%Infine ho messo in evidenza come in un serie di due ShuntLDO le fluttuazioni che si hanno sulla catena non sono viste dal carico applicato allo stesso.


%Lo studio di questo prototipo ha confermato l'affidabilità del principio di funzionamento con cui è stato progettato lo ShuntLDO, ha evidenziato gli aspetti cruciali per avere un bilanciamento delle correnti durante l'accensione e ha dimostrato la capacità di isolare il carico dai disturbi esterni presenti sulla linea di alimentazione.
%Allo stesso modo sono state messe in luce criticità e problemi dell'implementazione dello ShuntLDO all'interno del ROC RD53A, tutt'ora sotto studio. Tra queste spiccano l'importanza di un circuito che generi le tensioni di riferimento in modo affidabile, la presenza di resistenze spurie e una dipendenza del valore dell'offset dal valore della resistenza caratteristica. 
